\documentclass[12pt]{article}
\usepackage[utf8]{inputenc}
\usepackage[backend=biber,style=chicago-authordate,uniquename=false,maxbibnames = 99]{biblatex}
	\bibliography{Political Alienation.bib}
\usepackage{multicol}
\usepackage{ntheorem}
	\theoremseparator{:}
	\newtheorem{hyp}{Hypothesis}
\usepackage{lscape}
\usepackage{lipsum}
\usepackage{amsmath}
\usepackage{etoolbox}
\AtBeginEnvironment{quote}{\singlespacing\small}
\usepackage{multirow}
\usepackage{fnpct}
\usepackage{fancyhdr}
\usepackage{caption}
\usepackage{adjustbox}
\usepackage{hyperref}
\usepackage{array}
\usepackage[title]{appendix}
\usepackage{float}
\usepackage{subcaption}
	\captionsetup{belowskip=12pt,aboveskip=4pt}
\usepackage{cleveref}
\usepackage{graphicx}
\usepackage{threeparttable}
\usepackage{tablefootnote}
\usepackage{setspace}
	\interfootnotelinepenalty=10000
\usepackage{fullpage}
\usepackage{geometry}
    \geometry{top=1in, bottom = 1in, left = 1in, right = 1in}
\usepackage{endnotes}
    \doublespacing
    \setlength{\parindent}{1cm}
	\setlength{\headsep}{0.3in}
\newcommand{\pkg}[1]{{\fontseries{b}\selectfont #1}}
\makeatletter
\newcounter{subhyp} 
\let\savedc@hyp\c@hyp
\newenvironment{subhyp}
 {%
  \setcounter{subhyp}{0}%
  \stepcounter{hyp}%
  \edef\saved@hyp{\thehyp}% Save the current value of hyp
  \let\c@hyp\c@subhyp     % Now hyp is subhyp
  \renewcommand{\thehyp}{\saved@hyp\alph{hyp}}%
 }
 {}
\newcommand{\normhyp}{%
  \let\c@hyp\savedc@hyp % revert to the old one
  \renewcommand\thehyp{\arabic{hyp}}%
} 
\makeatother





%%%%PAPER INFO%%%%
\title{Alie(n)ation: Political Outsiders in the 2016 U.S. Presidential Election}
\author{Maxwell B. Allamong \thanks{Ph.D. Candidate of Political Science, Texas A\&M University, allamong@tamu.edu
} }
\date{\today}



%%%%SPACING, START DOC, TITLE%%%%

\begin{document}
\maketitle
\footnotetext[1]{Replication materials are available \href{https://github.com/maxallamong/Alienation}{\underline{here}}.}
\thispagestyle{empty}
\doublespacing





 %%%% ABSTRACT %%%%
\begin{abstract}  
The 2016 U.S. presidential election was noteworthy in that it featured so-called ``political outsiders" on both major parties' primary tickets. Donald Trump and Bernie Sanders, neither of whom held formal leadership positions within their party, found surprising amounts of success, with Trump eventually winning the presidency. What explains the ability of these unconventional candidates to capture such broad support? I argue that Trump and Sanders secured ``protest votes" from those feeling disaffected from the political system, also known as the politically alienated. Combining open-ended responses about outsider candidates with newly developed text-analysis tools, I show that those alienated from both the inputs and outputs of the political system were more likely to say they liked Trump and Sanders for being outsiders. Then, I show that output-based alienation increased the Trump and Sanders vote, while also having a uniquely positive effect on turnout in 2016 compared to previous elections.


%Political alienation describes a feeling of estrangement from the political system. While some have suggested that political alienation can lead to protest voting in national elections---that is, voting \textit{against} a particular candidate or entity---the mechanism through which alienation affects such behaviors remains unclear. When and why might the politically alienated cast protest votes and how do we know those votes signal a rejection of the political system? I argue that the politically alienated can channel their feelings of \textit{political cycnism} and \textit{election inefficacy} through their support for `political outsiders.' Combining open-ended responses about outsider candidates with newly developed text-analysis tools, I show that those harboring feelings of cynicism and election inefficacy were more likely to `like' Trump and Sanders for being outsiders. I also show that cynicism was positively related to voting for Trump and Sanders, and that alienation had a unique motivating effect on turnout as compared to previous elections that featured prominent outsiders (e.g., Perot in 1992). 
%cynicism increased the likelihood of turning out to vote for outsiders like Donald Trump and Bernie Sanders, and that the vote-suppressing effect of election inefficacy on turnout was weakened in 2016. 
\end{abstract}\clearpage\pagenumbering{arabic}










%%%%%%%%%%%%%%%%%%%%%%%%%%%%%%%%%%%%%%%%%
%%%%%%%%%%%%%%%%%%%%%%%%%%%%%%%%%%%%%%%%%
%%                                     %%
%%   %   %    %  %%%%%  %%%%%  %%%%%   %%
%%   %   % %  %    %    %   %  %   %   %%  
%%   %   %  % %    %    %%%%%  %   %   %%
%%   %   %   %%    %    %  %   %   %   %%
%%   %   %    %    %    %   %  %%%%%   %%
%%                                     %%
%%%%%%%%%%%%%%%%%%%%%%%%%%%%%%%%%%%%%%%%%
%%%%%%%%%%%%%%%%%%%%%%%%%%%%%%%%%%%%%%%%%

\section{Introduction}
% What is the puzzle?
Presidential election cycles in the United States often follow a familiar pattern: Democratic and Republican politicians with the greatest name recognition jump into their party's primary, seeking not only the support of voters, but also the blessings of prominent party leaders.\footnote{This is, more or less, the central argument in \citetitle{cohen2009party} \parencite{cohen2009party}.}  The winners of these primary elections are typically established political figures, often having held office at either the state or federal level and having demonstrated service and loyalty to their party. The 2016 election cycle broke from this tradition, however, in that it featured so-called ``political outsiders" on both sides of the aisle: Donald Trump and Bernie Sanders. Neither candidate had previously held a leadership position within their party, with Sanders having been one of the few Independents in the U.S. Senate and Trump having never occupied an elected office. And yet, both Trump and Sanders received a substantial proportion of the primary vote, and even more surprising is that Trump would go on to win the general election over his Democratic opponent, Hillary Clinton, whom many would consider the ultimate ``political insider."

% What is my explanation?
Recent scholarly efforts to identify the sources of Trump's and Sanders' support have mostly considered factors that fall along social and economic lines such as status threat \parencite{mutz2018status}, social identities \parencite{sides2018identity,mason2021activating}, or racial and anti-immigrant resentment \parencite{hooghe2018explaining,hopkins2021activation}. However, there has been less consideration of the role of negative attitudes towards our political structures in elevating these two political outsiders to national prominence. In this paper, I argue that Trump and Sanders were uniquely positioned to capture ``protest votes" \parencite{alvarez2018taxonomy,southwell1998electoral}---or votes cast \textit{against} a particular entity---from individuals that felt disaffected from the political system, also known as the ``politically alienated" \parencite{olsen1969}. I highlight two specific dimensions of political alienation, including \textit{input-based alienation}---or alienation from the inputs to the political system such as the electoral process---and \textit{output-based alienation}---or alienation from the outputs to the political system which is often characterized by distrust or cynicism directed toward the government. I argue that both dimensions of alienation are capable of influencing one's evaluations of the candidates, as well as their vote choice. Alienation on either dimension will make individuals attracted to candidates like Trump and Sanders specifically because they present a challenge to the political system, but only output-based alienation is expected to be related to support for outsiders at the ballot box. Input-based alienation, however, precludes the use of elections (a political system input) to signal discontent, making it unclear if the presence of outsiders in the race will be sufficient to prime turnout for those alienated on this dimension.
 
 % How will I know if i'm right?
 To empirically evaluate my argument, I rely on data from the 2016 American National Election Study (ANES) and the January 2016 wave of the Institute for the Study of Citizens and Politics' (ISCAP) panel study. I begin by using a semi-automated text-analysis approach---the Structural Topic Model \parencite{roberts2014structural}---to explore open-ended survey responses describing what people liked about Trump and Sanders. Topics emerge from the models that are directly related to Trump's and Sanders' statuses as political outsiders. The ability of the Structural Topic Model to estimate relationships between variables of interest and topic usage allows me to show that both input- and output-based measures of alienation increase the likelihood that people state a preference for Trump and Sanders due to their ``outsider" status.  Importantly, these relationships are robust to the inclusion of co-variates that are known predictors of Trump's or Sanders' support, such as ideology, sexism, attitudes towards social and racial groups, attitudes toward immigration, and authoritarianism \parencite{hooghe2018explaining,dyck2018primary,mason2021activating,sides2018identity,mutz2018status,knuckey2020authoritarianism}. Having established that political alienation shapes the way that people view outsider candidates, I then show that alienation also affected vote choice---those alienated on the output dimension were more likely to vote for Trump and Sanders in the 2016 election, while input-based alienation largely did not benefit these candidates. 

% What is the contribution/Why is this important?
There are two ways in which the findings presented here contribute to our understanding of the relationship between public opinion and voting behavior.  First, I unpack a mechanism underlying the protest vote. A protest vote is, by definition, a vote cast for a candidate as a means of signaling discontent with the political system, but no previous work has explicitly shown that the politically alienated think of candidates in this way. Through the use of open-ended responses, however, I show that the politically alienated did in fact see Trump and Sanders as vectors for voicing one's discontent, paving the way for a protest vote. Second, I demonstrate that political alienation played a significant role in Trump's and Sanders' electoral success, even when accounting for the factors that are already known to prime support for these candidates (e.g., racial and anti-immigrant animus, identity, status threat, etc.). These results suggest that a more complete understanding of the surprising success of these non-traditional candidates, and similar candidates that may emerge in the future, requires serious consideration of the role of political alienation. %Furthermore, the conclusions of this investigation will bear particular importance moving forward as popular indicators of political alienation \parencite[like political distrust,][]{Citrin2018} continue to show the public growing more disaffected. 

 
%Second, my results help to contextualize the role of political alienation as a contributing factor to the success of Trump and Sanders alongside factors that are already known to prime support for these candidates (e.g., racial and anti-immigrant animus, identity, status threat, etc.). A complete understanding of the surprising success of these non-traditional candidates, and similar candidates that may emerge in the future, requires an investigation into the role of political alienation. Furthermore, the conclusions of this investigation will bear particular importance moving forward as popular indicators of political alienation \parencite[like political distrust,][]{Citrin2018} continue to show the public growing more disaffected. 




%%%%%%%%%%%%%%%%%%%%%%%%%%%%%%%%%%%%%%%%%%%%%%%%%%%%%%%%%%%%%%%%%%%%%
%%%%%%%%%%%%%%%%%%%%%%%%%%%%%%%%%%%%%%%%%%%%%%%%%%%%%%%%%%%%%%%%%%%%%
%%                                                                 %%
%%   %      %  %%%%%     %%%%%  %%%%%  %   %   %  %%%%%  %  %  %   %%
%%   %      %    %       %   %  %       % %    %  %      %  %  %   %%
%%   %      %    %       %%%%%  %%%%%   % %    %  %%%%%  %  %  %   %%
%%   %      %    %       %  %   %        %     %  %      %  %  %   %%
%%   %%%%%  %    %       %   %  %%%%%    %     %  %%%%%  %%%%%%%   %%
%%                                                                 %%
%%%%%%%%%%%%%%%%%%%%%%%%%%%%%%%%%%%%%%%%%%%%%%%%%%%%%%%%%%%%%%%%%%%%%
%%%%%%%%%%%%%%%%%%%%%%%%%%%%%%%%%%%%%%%%%%%%%%%%%%%%%%%%%%%%%%%%%%%%%

\section{Political Alienation: Definition and Effects}
% Broadly define alienation, motivate the use of separate "dimensions"
What does it mean to be politically alienated? The definition given by \textcite[][3]{citrin1975} closely reflects the popular conceptualization of political alienation as a ``relatively enduring sense of estrangement from existing political institutions, values, and leaders." Typically, feelings of alienation are considered ``diffuse" \parencite{easton1965systems} in nature, meaning they stem from evaluations of the political system in the broadest sense, and not from evaluations of specific political actors or policies. This definition performs well in capturing the essence of political alienation, but the precise ways in which one is estranged from the political system, and how those feelings of estrangement might influence other political attitudes and behaviors, remain unclear. As such, a number of scholars have delineated the various modes, dimensions, or categories of alienation. 

% Define the two dimensions
Early work on political alienation often applied the typology of social psychologist Melvin \textcite{seeman1959}, who identified five different modes of alienation including \textit{powerlessness}, \textit{normlessness}, \textit{meaninglessness}, \textit{isolation}, and \textit{self-estrangement}. Scholars working from this typology often narrowed in on a single dimension and examined its effect on various political attitudes or behaviors, such as \textcite{horton1962powerlessness} who examined the influence of powerlessness on negative voting. Over time, however, inconsistencies in the operationalization of these five dimensions and a lack of theorizing about their unique effects on political behavior led most scholars to adopt a two-dimensional conceptualization of political alienation, which I will also apply here. The labels used to describe these dimensions have not remained consistent, so an additional contribution I make here is to connect the dots between previous works on alienation that have focused on similar theoretic concepts using different terminology. 

The first dimension relates to an individual's beliefs about their inability to use the political process to affect the ``inputs" \parencite{almond1963civic,easton1965systems} to the political system. This dimension of alienation encapsulates concepts such as ``(in)efficacy" \parencite{campbell1960american,aberbach1969alienation}, ``political powerlessness" \parencite{finifter1970dimensions}, and ``attitudes of incapability" \parencite{olsen1969}. To avoid using all of these synonymous terms interchangeably, I will simply refer to alienation on this dimension as \textit{input-based alienation}.\footnote{While the \textit{input-based} and \textit{output-based alienation} terminology is novel to this project, the conceptual distinction is not---for instance, \textcite{olsen1969} spoke of ``attitudes of incapability" and ``attitudes of discontentment" while \textcite{finifter1970dimensions} distinguishes between ``political powerlessness" and ``political normlessness". I have introduced this new terminology with the hopes of standardizing the language we use to describe these dimensions, and also to facilitate the discussion between previous scholars of alienation that have used different terminology.} An example of input-based alienation would be if an individual felt that elections were an ineffective mechanism for capturing the attention of politicians. Elections are one of, if not the, primary means of making one's views known to those in positions of power, and when one feels that this process is failing, feelings of (input-based) alienation are likely to ensue. 

The second dimension of political alienation relates to one's feelings of discontentment or cynicism directed at the ``outputs" of the political system \parencite{almond1963civic,easton1965systems}. This dimension encapsulates concepts such as ``(dis)trust" \parencite{aberbach1969alienation}, ``cynicism" \parencite{southwell1998electoral}, ``political normlessness" \parencite{finifter1970dimensions}, and ``attitudes of discontentment" \parencite{olsen1969}. To again avoid confusion, I simply refer to alienation on this dimension as \textit{output-based alienation}. An example of an individual that is alienated on the output dimension is one who feels that the government is untrustworthy and that politicians don't represent the best interests of the people. Importantly, it should be noted from this example that one's feelings of output-based alienation are directed toward the government outputs \textit{as a whole}, and not toward specific policies like healthcare or tax reform. Of course it is possible that government failure on issues such as these can contribute to output-based alienation, but I am theoretically and empirically interested in broader feelings of alienaiton. 


% How does each dimension affect attitudes?
Broadly speaking, the primary way in which feelings of alienation are known to influence one's political attitudes is that they produce a sense of ``negativism" \parencite{horton1962powerlessness}. For instance, \textcite{thompson1960} found that the politically alienated were more likely to hold unfavorable views toward a local school bond referendum.\footnote{Both \textcite{thompson1960} and \textcite{citrin1975} use indexes of alienation that tap into both input- and output-based alienation, but neither set of authors explore the bi-dimensionality of alienation when assessing it's role in promoting negativism.} \textcite{citrin1975} also showed that the politically alienated held more negative evaluations of the current political climate and were more willing to support systemic change. This is to say that one's feelings of political alienation from the broader political system are known to spillover into one's attitudes toward more specific objects in the political environment, often casting them in a negative light. 

% How does each dimension affect behaviors?
The distinction between input- and output-based alienation becomes important when we consider their effects on political action. On the one hand, alienation from the inputs of the political system often appear negatively related to several forms of political participation such as voting \parencite{horton1962powerlessness,aberbach1969alienation,southwell1998electoral} and discussing politics with others \parencite{olsen1969,finifter1970dimensions}.\footnote{As previously mentioned, input-based alienation is inversely related to political efficacy, which scholars sometimes break down further into internal and external efficacy. Internal efficacy refers to an individual's beliefs about their own ability to influence the political system, while external efficacy refers to an individual's beliefs about the ability of our political institutions to be influenced by society more broadly \parencites{southwell1998electoral}. I do not have specific theoretic expectations for each of these sub-components of efficacy, but I do note that authors often posit similar expectations for either component \parencite{fox2020political,southwell1998electoral}. Refer also to fn 7.} Given that input alienation is the belief that one is incapable of influencing what goes into the political system, it is unsurprising that those alienated on this dimension would not often use the political process to air their grievances. On the other hand, alienation from the system's outputs appears typically unrelated to political participation \parencite{finifter1970dimensions,olsen1969}, as those harboring such feelings may or may not see the political process as a viable mechanism for signaling their discontent. These two dimensions of alienation, though they are known to have a similar negative effect on political attitudes, appear to have unique effects on political behaviors.

% What role does alienation play in elections?
The tendency for alienation to produce a sense of negativism is insightful in its own right, but it is especially relevant when we consider the role of alienation in elections. This is because the negativism that characterizes alienation may influence how one chooses to vote. One possibility explored in the literature is that alienation can increase the chances of casting ``negative" or ``protest" votes, which are votes cast with the intention of signaling one's discontent. For instance, several early studies of political alienation examined the effects of alienation in the context of local referenda where, unlike typical elections for office, voters are given the option to explicitly vote against a particular measure \parencite{mcdill1962status,horton1962powerlessness,thompson1960}. These studies were consistent in their finding that the politically alienated were disproportionately more likely to vote against the referendum. In the more traditional election setting, scholars such as \textcite{aberbach1969alienation} and \textcite{southwell1998electoral} have argued that the politically alienated cast protest votes for the insurgent candidacies of Barry Goldwater in 1964 and Ross Perot in 1992 (respectively), though neither of these works provides evidence to indicate that these candidates' insurgent status was a conscious consideration of the politically alienated when deciding who to vote for. More recently, alienation was suspected as a possible explanation for the success of the Brexit movement in the United Kingdom, though \textcite{fox2020political} found that political alienation had only a weak relationship with support for the movement.

% What do we still not know?
What remains unclear from the literature is an indication of \textit{how} alienation might produce a sense of negativism in the context of U.S. national elections. How can we be certain that a vote cast for an outsider candidate is meant as a sign of protest without first exploring the considerations motivating the vote? In the next section, I will argue that the candidacies of Donald Trump and Bernie Sanders in the 2016 election provided a unique opportunity to explore the mechanism behind protest voting in U.S. national elections. 


% Previously, scholars such as \textcite{aberbach1969alienation} and \textcite{southwell1998electoral} have argued that the insurgent candidacies of Barry Goldwater in 1964 and Ross Perot in 1992 (respectively) were bolstered by the politically alienated, but neither of these works provides evidence to indicate that these candidates' insurgent status was a conscious consideration of the politically alienated when deciding who to vote for. 

%\footnote{While I am not the first to distinguish between the two dimensions of alienation (e.g., \citeauthor{olsen1969}'s (\citeyear{olsen1969}) `attitudes of incapability' and `attitudes of discontentment,'



%%%%%%%%%%%%%%%%%%%%%%%%%%%%%%%%%%%%%%%%%%%%%%%%%%
%%%%%%%%%%%%%%%%%%%%%%%%%%%%%%%%%%%%%%%%%%%%%%%%%%
%%                                              %%
%%   %%%%%  %   %  %%%%  %%%%%   %%%%%  %   %   %%
%%     %    %   %  %     %   %   %   %   % %    %%
%%     %    %%%%%  %%    %   %   %%%%%    %     %%
%%     %    %   %  %     %   %   %  %     %     %%
%%     %    %   %  %%%%  %%%%%   %   %    %     %%
%%                                              %%
%%%%%%%%%%%%%%%%%%%%%%%%%%%%%%%%%%%%%%%%%%%%%%%%%%
%%%%%%%%%%%%%%%%%%%%%%%%%%%%%%%%%%%%%%%%%%%%%%%%%%

\section{Alienation and Outsiders in the 2016 Election}\label{sec:theory}
As \textcite{templeton1966} noted long ago, most typical presidential elections feature establishment-type candidates from either party, and the debates tend to center around prominent political issues of the day. In these elections, feelings of alienation are likely to play only a minor role: input-based alienation may dampen participation in the electoral process as it's known to do in other political contexts, while feelings of output-based alienation may take a backseat to partisan or ideological considerations \parencite{finifter1970dimensions}. However, the 2016 election deviated from this pattern as both major parties' primary elections featured so-called political outsiders. Donald Trump, a New York businessman with no prior office-holding experience, infiltrated the ranks of the Republican Party and would go on to win the presidency over the Democratic candidate, Hillary Clinton. Bernie Sanders, as one of only a handful of independents to ever hold a seat in the U.S. Senate, put up a serious fight in the 2016 Democratic primary. What role did political alienation play in elevating these candidates to national prominence?

I argue that the politically alienated were attracted to Trump and Sanders in the 2016 election due to their ``political outsider" personas, thus paving the way for these candidates to capture protest votes. Throughout the campaign, both candidates made explicit appeals to those feeling disaffected from the political system. Consider the following statement from Trump who is tapping into the feelings of output-based alienation when speaking at a campaign rally in Sioux City, Iowa \parencite{Jackson2016}:

\begin{quote}
	At the heart of this election is a simple question: will our country be governed by the people or will it be governed by the corrupt political class?
\end{quote}

\noindent This rhetoric sounds very much the same as the rhetoric of Bernie Sanders, who said the following at the Brookings Institution the same day he announced his intention to seek the Democratic nomination \parencite{Dews2015}:

\begin{quote}
	There is a lot of sentiment that enough is enough, that we need fundamental changes, that the establishment -- whether it is the economic establishment, the political establishment, or the media establishment -- is failing the American people.
\end{quote}

\noindent The sort of ``negativism" embodied in these statements---that the political system is corrupted and failing---should resonate most with those that feel alienated. For this reason we should expect political alienation---be it input- or output-based---to be related to the belief that Trump and Sanders are preferable due to their outsider status. 

Political alienation may shape attitudes towards certain presidential candidates, but was it also a driver of vote choice? I argue that political alienation can motivate individuals to cast protest votes for political outsiders, but that this process occurs primarily through feelings of output-based alienation. The specific type of protest vote that I am considering here is referred to by \textcite{alvarez2018taxonomy} as an ``insurgency party protest voting" and it describes the act of voting for fringe, or ``insurgent," parties or candidates as a means of signaling disaffection with other aspects of the political system (e.g., mainstream political parties). In the 2016 election, Trump and Sanders were clearly the insurgent candidates in the race and the quotes provided above indicate that these candidates clearly saw themselves as opponents of the political establishment. If I can show that feelings of alienation shaped how people viewed Trump and Sanders (i.e., seeing them as political outsiders), and also show that alienation predicts the Trump and Sanders vote, this would be highly indicative of a protest vote. 

Why might input- and output-based alienation have different effects on the likelihood of protest voting? I begin by considering the potential role of input-based alienation. As noted earlier, there is evidence to suggest that input-based alienation can discourage participation in the political process \parencite[e.g.,][]{aberbach1969alienation}. The relationship here is straightforward---one is not likely to participate if they feel distant from the input mechanisms. In the case of the 2016 election, then, this might suggest that the presence of outsiders in the race would be insufficient for those with input-based alienation to set aside their lack of faith in the political system's input mechanism in order to cast a protest vote. However, it is also possible that the unique circumstances of the 2016 election reshaped the relationship between participation and input-based alienation. As \textcite{southwell1998electoral} note, U.S. national elections rarely give people the chance to vote for a candidate that represents an opposition to the political system. Perot's third-party bid in the 1992 presidential election is the closest example in recent decades, at least until Trump and Sanders emerged in 2016. The prospect of voting for a political outsider, especially those that have infiltrated the ranks of major parties, may have provided to needed incentive for those with input-based alienation to cast a protest vote. 

Compared to input-based alienation, alienation from the political system's outputs is more readily expressed through one's political behaviors. The relationship here is also straightforward---if one disapproves of what the system produces, the remedy is to try to adjust the system to provide more favorable outcomes. In the context of national elections, this may entail voting for candidates that appear likely to disrupt the current political order if elected \parencite{aberbach1969alienation,southwell1998electoral}, though again there is no available evidence to suggest that the politically alienated consciously consider a candidate's outsider status before casting their vote. As the quotes above appear to indicate, however, the candidacies of Donald Trump and Bernie Sanders in 2016 were centered on the idea that they would serve as that disruptive force by taking on the ``corrupt political class" or the failing ``political establishment," positioning both candidates to benefit from protest votes. This should lead us to expect those with output-based alienation in 2016 to be motivated to turnout and vote for either Trump or Sanders. 

%For this reason, previous scholars have argued that the candidacies of Barry Goldwater in 1964 \parencite{aberbach1969alienation} and Ross Perot in 1992 \parencite{southwell1998electoral} provided such opportunities for the politically alienated, but in neither case was there evidence presented to indicate that politically alienated consciously considered these candidates' potential for disruption when deciding for whom to vote. However, the insurgent candidacies of Trump and Sanders in the 2016 election presents a new opportunity to directly uncover the motivations of the politically alienated. 

%both Barry Goldwater \parencite{aberbach1969alienation} and Ross Perot \parencite{southwell1998electoral} received substantially support from those with output-based alienation, although their candidacies do not appear to have boosted turnout. In the 2016 election, however, the insurgent candidacies of Trump and Sanders presented the most serious challenge to the political system in decades, sending a clear signal to those with output-based alienation that a vote for them was a vote against the prevailing political order. 

%The language of rational choice is useful here. Alienated individuals can be thought of placing a lower value on the continuation of our current democratic system, translating to a smaller ``D" term in the \citeauthor{downs1957economic}-ian (\citeyear{downs1957economic}) \parencite{southwell1998electoral}. 

%Trump and Sanders were positioned to secure ``protest votes" from the politically alienated. A protest vote is a vote that is cast with the specific intention of signaling discontent with (some aspect of) the political system. Through the ``outsider" personas that Trump and Sanders projected on the campaign trail, I believe that these candidates were successful at signaling to the politically alienated that a vote for them was a vote against the ``establishment." For example, consider the following statement from Trump who is tapping into the feelings of output-based alienation when speaking at a campaign rally in Sioux City, Iowa \parencite{Jackson2016}:



%%%%%%%%%%%%%%%%%%%%%%%%%%%%%%%%%%%%%
%%%%%%%%%%%%%%%%%%%%%%%%%%%%%%%%%%%%%
%%                                 %%
%%   %%%%    %%%%%  %%%%%  %%%%%   %%       
%%   %   %   %   %    %    %   %   %%
%%   %   %   %%%%%    %    %%%%%   %%
%%   %   %   %   %    %    %   %   %%       
%%   %%%%    %   %    %    %   %   %%
%%                                 %%
%%%%%%%%%%%%%%%%%%%%%%%%%%%%%%%%%%%%%
%%%%%%%%%%%%%%%%%%%%%%%%%%%%%%%%%%%%%

\section{Data and Methods}\label{sec:datamethods}
To examine the effects of political alienation in the context of the 2016 U.S. Presidential Election, I primarily rely upon data from the American National Election Studies (ANES). The ANES data are particularly well-suited for my purposes as they contain items that capture several dimensions of political alienation \parencite{mason1985}, as well as information on vote choice, party affiliation, attitudes on topics like immigration and race, and various demographics. The ANES data also include open-ended responses about the things that respondents `like' about the two major parties' nominees for president, allowing me to examine whether feelings of alienation shaped individual's stated reasons for liking Trump (specifically, that he is an outsider). Unfortunately, the ANES does not include open-ended questions about the candidates in the primary elections (e.g., Bernie Sanders), but such questions were asked in the January 2016 wave of the Institute for the Study of Citizens and Politics (ISCAP) panel study.\footnote{The Institute for the Study of Citizens and Politics (linked \href{https://www.asc.upenn.edu/research/centers/institute-for-the-study-of-citizens-and-politics}{\underline{here}}) is located at the University of Pennsylvania and has been conducting a panel study of American adults since 2012. The data used in this analysis come from wave 10 (January 2016) of this population-based online panel.} Therefore, I use the ISCAP data to examine how alienation may have also shaped perceptions of Bernie Sanders as a political outsider (and thus a potential vector for a protest vote). The remainder of this section will focus on describing these data in greater detail, along with my approach to model them.  

%%% MEASURES %%%
\subsection{Measures of Alienation}
From the ANES data, I operationalize input-based and output-based alienation using measures of \textit{electoral inefficacy} and \textit{cynicism}, respectively.\footnote{From the ANES data, my measures of input-based (\textit{electoral inefficacy}) and output-based (\textit{cynicism}) alienation demonstrate only a weak correlation ($r = 0.28$).} The first, \textit{electoral inefficacy}, is a measure of input-based alienation and it comes from a single item that asks, ``How much do you feel that having elections makes the government pay attention to what people think?" to which individuals may respond (0) ``A good deal," (1) ``Some," or (2) ``Not much." This question captures alienation from the inputs to the political system specifically as it relates to elections. While others have used more general measures of political efficacy when operationalizing output alienation \parencite[e.g.,][]{aberbach1969alienation}, the \textit{electoral inefficacy} item is appropriately focused on the context in which I expect alienation to have an effect (i.e., elections).\footnote{Unfortunately, the traditional `internal' and `external' efficacy items used by others to measure output alienation \parencite[e.g.,][]{fox2020political,southwell1998electoral} were measured post-election, where as the `cynicism' and `electoral inefficacy' items that I employ were measured pre-election. I choose to rely solely upon pre-election measures of alienation to avoid issues of time-dependency (i.e., levels of alienation being affected by the outcome of the election).} I have rescaled this variable to range between 0 (electorally empowered) and 1 (electorally inefficacious).  

The second variable, \textit{cynicism}, is a measure of output-based alienation and is derived from the `No Trust' and `Big Interests' items that are part of the ANES `Cynicism' index \parencite{mason1985}. The `No Trust' item asks, ``How often can you trust the federal government in Washington to do what is right?" and the `Big Interests' item asks, ``Would you say the government is pretty much run by a few big interests looking out for themselves or that it is run for the benefit of all people?" Answers to these questions are combined to form a scale that ranges from 0 (not at all cynical) to 1 (completely cynical).\footnote{Possible answers to the `No Trust' item include: (1) ``Always," (2) ``Most of the time," (3) ``About half the time," (4) ``\textit{Some of the time}," or (5) ``\textit{Never}." Answers to the `Big Interests' item include: (0) ``For the benefit of all people" or (1) ``\textit{Run by a few big interests}." To form the \textit{cynicism} scale, respondents are given a point for each cynical answer (italicized) that they provide, creating an initial measure that ranges from 0 (No cynical answers) to 2 (All cynical answers), which I then rescale to range between 0 and 1. The Spearman-Brown reliability coefficient for these two items is 0.54.} From their analysis of the various measures of political alienation contained in the ANES, \textcite{mason1985} conclude that the two items I am using here form ``a single latent construct reflecting a lack of trust in the ability of the federal government to act in ways that people regard as right and fair." (p. 145). From this definition, it is clear that this measure of \textit{cynicism} reflects alienation from the outputs of the political system. 

Although the ISCAP panel is somewhat limited in the number of measures of alienation it contains, there are two measures that I will use when analyzing the relationship between alienation and perceptions of Sanders as an outsider, including \textit{electoral inefficacy} and \textit{political system illegitimacy}.\footnote{From the ISCAP data, my measures of \textit{electoral inefficacy} and \textit{political system illegitimacy} are only weakly correlated ($r = 0.27$).} The first, electoral inefficacy, is the same as the ANES measure of electoral inefficacy described above. The second, political system illegitimacy, taps into feelings of diffuse support for our current governing system. This measure asks respondents to state how much they agree or disagree with the four following statements:

\singlespacing
\begin{enumerate}
	\item I would rather live under our system of government than any other that I can think of.
	\item Our system of government is in need of some serious changes.
	\item Whatever its faults may be, our form of government is best for representing the interest of the country's citizens.
	\item At present I feel very critical of our political system.
\end{enumerate}
\doublespacing

While this measure clearly captures alienation from the political system, it is not immediately clear if it is tapping into alienation from the inputs or the outputs. For example, agreeing that the political system needs serious changes (Statement 2) does not make clear if it is the system's inputs, outputs, or both that need changing. I operate under the assumption that these statements tap into both dimensions and use principal components analysis to create a single index that ranges from 0 to 1, with higher values representing stronger beliefs that the political system is illegitimate.\footnote{More information on this index can be found in Appendix~\ref{app:legitimacy}.} 

\subsection{Hypotheses}
My argument regarding the effects of political alienation in national elections leads naturally to several expectations. First, I expect that both input-based and output-based measures of alienation will increase the likelihood that a respondent likes Trump or Sanders for their outsider qualities. From the ANES responses about Trump, this implies that both \textit{electoral inefficacy} (an input-based measure) and \textit{cynicism} (an output-based measure) should be positively related to the view of Trump as an outsider (Hypothesis~\ref{hyp:candevals-trump}). From the ISCAP responses about Sanders, both \textit{electoral inefficacy} (an input-based measure) and \textit{political system illegitimacy} (a measure of both dimensions of alienation) should be positively related to the view of Sanders as an outsider (Hypothesis~\ref{hyp:candevals-sanders}).

 %As I have argued, candidates that present themselves as a challenge to the political system are likely to be attractive to those that feel alienated from that same system. In the context of the 2016 election, this expectation should apply to both Trump and Sanders.\footnote{As previously stated, I assume that the political system illegitimacy index from the ISCAP panel taps into both input- and output-based alienation. With respect to Hypothesis~\ref{hyp:candevals}, then, the specific expectation with respect to this measure is that political system illegitimacy will increase the likelihood of liking Sanders because he is a political outsider.}

% 1
\begin{hyp}\label{hyp:candevals-trump}
Both Electoral Inefficacy and Cynicism should increase the likelihood of liking Trump because he is an outsider
\end{hyp}

\begin{hyp}\label{hyp:candevals-sanders}
Both Electoral Inefficacy and Political System Illegitimacy should increase the likelihood of liking Sanders because he is an outsider
\end{hyp}

%\begin{hyp}\label{hyp:candevals}
%Both input-based and output-based alienation increase the likelihood of liking Trump or Sanders because they are political outsiders
%\end{hyp}

Next, I consider the effect of input-based measures of alienation on voting behavior. On one hand, I might expect input-based measures---specifically \textit{electoral inefficacy} from the ANES data---to promote abstention in the 2016 election (Hypothesis~\ref{hyp:input-abstain}), as those that feel alienated from the inputs to the political system may avoid using those mechanisms (e.g., elections) to signal their discontent, even in the presence of political outsider candidates. So although input-alienation may lead to a stated preference for outsiders, that preference may not manifest in vote choice. On the other hand, it may be the case that those with input-based alienation (i.e., the electorally inefficacious) are particularly inclined to turn out and vote for the outsiders Trump and Sanders (Hypothesis~\ref{hyp:input-vote}), as these individuals have the most to gain from seeing outsiders win and follow through on their promises to upend the political system. Such a finding would clearly be at odds with previous literature showing that input-based alienation depresses political participation, but would provide valuable insight into the mechanism behind the protest vote. From this, I am led to propose two hypotheses:

\begin{subhyp}
\begin{hyp}\label{hyp:input-abstain}
	Electoral Inefficacy increases the likelihood of abstention in the 2016 primary and general elections
\end{hyp}
%	
\begin{hyp}\label{hyp:input-vote}
	Electoral Inefficacy increases the likelihood of turning out to vote for Sanders and Trump
\end{hyp}	
\end{subhyp}

 
 % 2
%\begin{hyp}\label{hyp:input-voting}
%Input-based alienation is negatively related to turnout in typical national elections, and unrelated to turnout and vote choice in the 2016 election
%\end{hyp}

Finally, I expect that output-based measures of alienation---specifically \textit{cynicism} from the ANES data---will increase the likelihood that individuals turn out to vote for Sanders in the Democratic primary and for Trump in the Republican primary and general election (Hypothesis~\ref{hyp:output-voting}). As scholars have noted, the two major parties' candidates for presidents are typically establishment figures \parencite{templeton1966,cohen2009party}, but with Trump and Sanders in the race, those that were unsatisfied with the outputs of our political system were given a rare opportunity to use the national election process to signal their disaffection. If those with feelings of output-based alienation are indeed capitalizing on this opportunity, I expect this form of alienation to be directly related to voting behavior in terms of turnout and vote choice. This leads to my final hypothesis:

% 3
\begin{hyp}\label{hyp:output-voting}
Cynicism increases the likelihood of turning out and voting for Sanders in the Democratic primary and for Trump in the Republican primary and general election
\end{hyp}



%and \textit{election unresponsiveness} will have distinct effects on voting behavior: \textit{cynicism} will increase the likelihood that individuals turn out to vote for Sanders in the Democratic primary and for Trump in the Republican primary and general election (Hypothesis~\ref{hyp:cyn-voting}), while \textit{election unresponsiveness} will decrease the likelihood that individuals turn out (Hypothesis~\ref{hyp:elect-voting}). As scholars have noted \parencite{templeton1966}, the two major parties' candidates for presidents are typically establishment figures. With Trump and Sanders in the race, those that were unsatisfied with the outputs \parencite{almond1963civic,easton1965systems} of our political system---that is, those holding attitudes of \textit{discontentment}---were given a rare opportunity to use the national election process to signal their disaffection. Those with attitudes of \textit{incapability}, on the other hand, are no doubt expected to admire the anti-establishment qualities of candidates like Trump (Hypothesis~\ref{hyp:candevals}), but their belief that elections are ineffective tools for sparking political change means that we shouldn't necessarily expect them to be \textit{more} likely than the politically capable to turnout in the 2016 election. Instead, I expect that \textit{incapability} will be negatively related to turnout in typical national elections, but that this relationship will weaken in 2016 due to the presence of `political outsiders' in the race. 

%However, one's attitudes towards a candidate are not necessarily channeled into political behaviors. This may be especially true of the politically alienated, as those harboring attitudes of \textit{incapability}, by definition, feel as though the political structures typically used to voice one's concerns are ineffective. Therefore, I exp

%There are three hypotheses that follow my argument that the presence of political outsiders in the 2016 election provided the politically alienated with an opportunity to voice their discontent for the political system. First, I expect that the alienated tended to vote at lower rates in previous national elections, but that this relationship will weaken in the 2016 election. While the alienated may have seen less value in turning out in previous elections, the presence of political outsiders should have motivated them to vote in 2016. Second, I expect that the politically alienated are attracted to candidates like Trump and Sanders specifically because they are political outsiders. When asked what they like about these candidates, the politically alienated should state that they see these candidates as presenting a challenge to the political system. Third and finally, I expect that the politically alienated will make their voices heard by voting for Sanders in the Democratic primary, and Trump in the Republican primary and general election.

%%% OPEN-ENDED LIKES ABOUT TRUMP, STRUCTURAL TOPIC MODEL %%%
\subsection{Open-Ended Responses and the Structural Topic Model}
To understand whether the politically alienated were more likely to state a preference for Trump or Sanders due to their outsider statuses (Hypotheses~\ref{hyp:candevals-trump} and \ref{hyp:candevals-sanders}), I rely on open-ended responses about these two candidates from the ANES pre-election survey taken during the general election campaign and the ISCAP panel study.\footnote{The redacted ANES open-ended responses used in this analysis are publicly available from the organization's web page (https://electionstudies.org).} The open-ended question from the ANES that I am interested in asked all respondents, ``Is there anything in particular about Donald Trump that might make you want to vote for him?" If respondents provide a first thought, the interview follows up by asking ``anything else?" until the respondent provides up to five mentions or says ``no."\footnote{For each respondent, their initial response and all follow-ups to the interviewers prime are contained in a single document (or cell) per respondent. There is no unique identifier to separate each respondents `likes' into different cells, so all mentions from a respondent must be analyzed together. Respondents that provided no `likes' about Trump are coded as Not Applicable (NA), so their data cannot be used in the estimation of the Structural Topic Model.} Of the 4,270 respondents in the 2016 sample, approximately 1,853 respondents ($\sim$44\%) provided a response to this question, and of those, 1,099 respondents ($\sim$26\%) had complete co-variate data. Unfortunately, open-ended responses about Bernie Sanders are unavailable from the ANES as the open-ended questions were only asked about candidates in the general election (i.e., Trump and Clinton). 

Open-ended responses about Bernie Sanders, then, come from the January 2016 wave of the ISCAP panel study. In the survey, respondents that identified as either Democrat or Republican were asked which candidate they prefer in their party's primary election.\footnote{Respondents were not asked about candidates from the out-party's primary election.} The question that I am interested in specifically asks ``Let's say a friend asked you why you were supporting [Democrat/Republican candidate] in the primary election. In one sentence, what would you say?" Here I am limited to analyzing only Democratic respondents (including leaners) that initially stated a preference for Bernie Sanders in the Democratic primary. Of the 2,471 respondents in this wave, 785 (51\%) identified as Democrat and roughly a third ($\sim$32\%) of those Democrats preferred Sanders. After removing observations with incomplete covariate data, I am left with 174 respondents. 

The open-ended responses were provided from either source as raw text and had not been coded into discrete categories based on their content (as the ANES has done in the past).\footnote{Information on pre-processing these texts is given in Appendix~\ref{app:preprocessing}.} Fortunately, several forms of (semi-) automated content analysis have emerged to assist researchers in categorizing large bodies of text \parencite{grimmer2013text}. I use one such approach, the Structural Topic Model \parencite{roberts2019stm}, to assist me in categorizing these open-ended responses about Trump and Sanders. \textcite{roberts2014structural} has previously shown that the topics that emerge from a Structural Topic Model performed on open-ended responses are coherent and often mimic the categories assigned by human coders (e.g., the ANES `Most Important Problem in Washington' question). As my results show, the STM also performs quite well with the open-ended responses about Trump and Sanders. 

The intuition behind the Structural Topic Model is simple: identify clusters of words that tend to co-occur (i.e., topics). This is the same basic intuition underlying more common forms of topic models such as LDA \parencite{Blei2003}, but the STM is unique in the sense that it allows researchers to include covariates that they suspect will affect 1) the use of certain topics (\textit{prevalence}), or 2) the use of specific words within a topic (\textit{content}) \parencite{roberts2014structural}. My expectation is that feelings of political alienation will increase the frequency with which respondents use the ``political outsider" topic to describe Trump or Sanders, so I choose to specify cynicism and electoral inefficacy as prevalence covariates for the ANES responses about Trump, and specify political system illegitimacy and electoral inefficacy as prevalence covariates for the ISCAP responses about Sanders.\footnote{Partisanship is also included as a prevalence co-variate in the Trump model given the crucial role it plays in shaping many political behaviors, but is not specified as a prevalence co-variate in the Sanders model since only Democrats had the potential to give a response about Sanders.}

One advantage of the Structural Topic Model---compared to manual coding or supervised machine learning approaches---is that the researcher need not provide a set of `training' documents from which each topic should be built.\footnote{Supervised approaches to document classification are dependent on the researchers coding scheme.} Instead, the STM takes a `bottom-up' approach, allowing the machine to generate topics from the data. However, the STM does still require a small amount of supervision as the researcher must decide the number of topics (\textit{K}) that are to be found. \textcite{roberts2019stm} note that there isn't necessarily a universally correct number of topics for a given set of documents, and advise researchers to rely on substantive knowledge of the data and, if necessary, explore models that range in their number of topics and select the model that demonstrates favorable properties (e.g., high semantic coherence and exclusivity). Appendix~\ref{app:modelselection} contains a more thorough discussion of the process that I used select the number of topics, which led me to estimate a model with 27 topics for the responses about Trump and a model with 13 topics for the responses about Sanders. 

%The generally short length of the open-ended responses leads the STM to produce topics that are quite specific in content, so I place my focus on understanding which qualities of respondents drive the use (i.e., prevalence) of particular topics. The STM is also unique in the sense that it takes a `bottom-up' approach, allowing the machine to infer topics from the data, as opposed to a `top-down' or supervised approach that requires researchers to inform the machine which latent categories are present in the text. However, the STM does still require a small amount of supervision, as the researcher must decide the number of topics that are to be found. \textcite{roberts2019stm} note that there isn't necessarily a ``correct" number of topics for a given set of documents, and advise researchers to explore models that range in their number of topics, selecting one's model based on measures such as semantic coherence or exclusivity. Following this advice, I undertake an iterative process to narrow-in on a model with 26 topics.\footnote{Appendix~\ref{app:modelselection} provides a more detailed discussion of the process of selecting the number of topics.} In this model, I specify `cynicism' and `election unresponsiveness' as \textit{prevalence} covariates, as I expect these variables to alter the frequency with which one draws upon particular topics when stating what they `like' about Trump.\footnote{Figures~\ref{fig:exploringtopics} and \ref{fig:coef-stm} derive from an STM that included an additive relationship between prevalence covariates.} The model outputs are presented in the \nameref{sec:results} section. 

%%% MODELS OF VOTING BEHAVIOR %%%
\subsection{Models of Voting Behavior}
My examination of the effects of alienation on voting behavior in the 2016 election centers largely on the results of two models, both of which use ANES data. I begin by using a multinomial logit to model the effects of electoral inefficacy and cynicism in the 2016 primary elections. Here, I restrict my sample to respondents from states that hosted open primaries on Super Tuesday in 2016 and use vote choice as the dependent variable with the options being ``Sanders, ``Trump," ``Other," and ``Did Not Vote."\footnote{``Other" includes all candidates other than Trump or Sanders---regardless of partisanship---that ran in the 2016 primary elections.}\footnote{States hosting open primaries on Super Tuesday in 2016 include: Alabama, Arkansas, Georgia, Minnesota, Tennessee, Texas, Vermont, and Virginia.} Restricting my sample in this way allows me to avoid the possibility of unobserved time-dependencies that could arise from pooling respondents that voted at different points in the election cycle. Additionally, focusing on open primaries allows me to include both Democrats and Republicans in the same model, while also allowing for the possibility that some individuals voted for candidates of the out-party. This model includes electoral inefficacy and cynicism as the primary predictors alongside a range of co-variates that are known to influence support for Trump or Sanders such as: attitudes towards Democratically-aligned social groups including Blacks, Muslims, Hispanics, and LGBT \parencite{mason2021activating,hopkins2021activation}; indicators of status threat such as opposition to free-trade and beliefs about the military threat posed by China \parencite{mutz2018status}; authoritarian tendencies \parencite{knuckey2020authoritarianism}, racial and partisan identification \parencite{sides2018identity}; anti-immigrant attitudes \parencite{sides2018identity,hooghe2018explaining}; sexism \parencite{valentino2018mobilizing,sides2018identity}, Evangelical identity \parencite{margolis2020wants}, and a host of more common co-variates such as economic evaluations, political interest, ideology, income, education, sex (female), and age.\footnote{All variables rescaled to range between 0 and 1. See Appendix~\ref{app:variablecoding} for more information about the variables used in these analyses.}\footnote{Some of the co-variates included in these models were recorded after the general election, raising additional concerns about unobserved time dependencies. I present the primary and general election models with pre-election variables only in Tables~\ref{tab:primary-pre} and \ref{tab:general-pre} of Appendix~\ref{app:votemodels}. These models show little to no change from the fully specified models in the main text.}

 Next, I estimate a multinomial logit for the general election, where the dependent variable is once again vote choice, with the options being ``Clinton," ``Trump," ``Other," and ``Did Not Vote." This model includes the same primary predictors (electoral inefficacy and cynicism) and co-variates (listed above) as the model of the primary election. Unlike the model of the primary election, however, the sample for the general election is not limited to particular states. 
  
%(treated as categorical with `Independent' as the reference category, leaners coded as partisans), ideology (ranges from 1 = extremely liberal to 7 = extremely conservative), partisan strength (0 = pure independent to 3 = strong partisan), income (in quartiles), white (1 if white, 0 otherwise), female (1 if female, 0 otherwise), age (17-99 continuous), political interest (1 = not much interested to 3 = very much interested), and education (1 = less than high school to 5 = Bachelors or above).

%Finally, I estimate a series of logistic regressions to model the effects of \textit{cynicism} and \textit{electoral inefficacy} on turnout in the last eight elections (including 2016). This allows me to show that the effects of alienation I find in 2016 are largely unique to that particular election cycle. These models also include a number of co-variates that confound the relationship between either category of alienation and turnout, including.....










%%%%%%%%%%%%%%%%%%%%%%%%%%%%%%%%%%%%%%%%%%%%%%%%%%%%%%%%
%%%%%%%%%%%%%%%%%%%%%%%%%%%%%%%%%%%%%%%%%%%%%%%%%%%%%%%%
%%                                                    %%
%%   %%%%%  %%%%%  %%%%%  %   %  %     %%%%%  %%%%%   %%
%%   %   %  %      %      %   %  %       %    %       %%
%%   %%%%%  %%%%%  %%%%%  %   %  %       %    %%%%%   %% 
%%   %  %   %          %  %   %  %       %        %   %%
%%   %   %  %%%%%  %%%%%  %%%%%  %%%%%   %    %%%%%   %%
%%                                                    %%
%%%%%%%%%%%%%%%%%%%%%%%%%%%%%%%%%%%%%%%%%%%%%%%%%%%%%%%%
%%%%%%%%%%%%%%%%%%%%%%%%%%%%%%%%%%%%%%%%%%%%%%%%%%%%%%%%

\section{Results}\label{sec:results}
I begin by exploring the results of the Structural Topic Models. Figures~\ref{fig:top6topics-trump} and \ref{fig:top6topics-sanders} present the top 6 topics from the Trump and Sanders models, respectively, along with their expected topic proportions across all documents in their respective corpora.\footnote{Figures~\ref{fig:alltopics-trump} and \ref{fig:alltopics-sanders} in Appendix~\ref{app:stmresults} show the expected topic proportions for all topics.} I have assigned a label (light gray text) to each topic based on the words that are most closely associated with each topic, and also through an examination of documents that contain a high proportion of a particular topic. The topics that arise are both coherent and insightful---for instance, the topic most commonly mentioned by ANES respondents related to Trump's experience in business (expected topic proportion $\approx$ 0.06). For ISCAP respondents, the most prevalent topic about Sanders related to his caring nature towards others and his desire to represent their beliefs and interests (expected topic proportion $\approx$ 0.12). Clearly the STMs performed quite well at identifying the various themes that underlie Trump's and Sanders' support. 

% Topic Proportions, Top 8 Topics
\begin{figure}[t!]
\centering
   \begin{subfigure}[b]{0.75\textwidth}
	   \centering
	   \includegraphics[width=\textwidth]{Figures/top6topics-trump.pdf}
	   \caption{Trump}
	   \label{fig:top6topics-trump} 
	\end{subfigure}
\\ 
	\begin{subfigure}[b]{0.75\textwidth}
		\centering
	   \includegraphics[width=\textwidth]{Figures/top6topics-bernie.pdf}   
	   \caption{Sanders}
	   \label{fig:top6topics-sanders}
	\end{subfigure}
	\caption{Expected Topic Proportion for Top 6 Topics with FREX Words}\label{fig:top6topics}
	\vspace{-.25cm}
	{\scriptsize \textit{Note:} Researcher designated labels given in parentheses, FREX words are those that are both frequent and exclusive to a topic}
\end{figure}

In this analysis, the topics that are of particular interest are those labeled ``Political Outsider."\footnote{In Appendix~\ref{app:model-responses-other-candidates}, I show that an ``outsider" topic does not arise when we explore the open-ended responses about other candidates in the 2016 presidential election race including Hillary Clinton (Figure~\ref{fig:alltopics-clinton}) and the numerous candidates in the Republican primary (Figure~\ref{fig:alltopics-reps}).} In the Trump model, the outsider topic is expected to account for roughly 5\% of the average response, whereas in the Sanders model the outsider topic is expected to account for roughly 10\% of the average response. The word stems associated with these topics in Figure~\ref{fig:top6topics} indicate that part of Trump's and Sanders' appeal was due to their perceived disassociation from politics and the political system. For Trump, the political outsider topic is characterized by such words as ``politician," ``career," ``trustworthi," and ``interest." For Sanders, words such as ``fight," ``corpor(ate/ation)," ``influence," and ``money" are used often in this topic. While there is no doubt that the words associated with these outsiders topics are suggestive, they are not fully revealing. Therefore, I have also provided several verbatim responses in Figures~\ref{fig:exemplartexts-trump} and \ref{fig:exemplartexts-bernie} that, according to the models, have dedicated a large proportion of their content to the outsider topic. 

% Exemplary Texts
\begin{figure}[thp!]
\centering
   \begin{subfigure}[b]{0.45\textwidth}
	   \centering
	   \includegraphics[width=\textwidth]{Figures/exemplartexts-trump.pdf}
	   \caption{{\small Trump - Political Outsider Topic}}
	   \label{fig:exemplartexts-trump} 
	\end{subfigure}
	\begin{subfigure}[b]{0.45\textwidth}
		\centering
	   \includegraphics[width=\textwidth]{Figures/exemplartexts-bernie.pdf}
	   \caption{{\small Sanders - Political Outsider Topic}}
	   \label{fig:exemplartexts-bernie}
	\end{subfigure}
	\caption{Exemplary Texts from Political Outsider Topics}\label{fig:exemplartexts}
	\vspace{-.25cm}
	%{\footnotesize \textit{Note:} Researcher designated labels given in parentheses, FREX words are those that are both frequent and exclusive to a topic}
\end{figure}

Looking first at Figure~\ref{fig:exemplartexts-trump}, the responses indicate that Trump was liked specifically because he lacked political experience. Some respondents saw it as a positive that Trump was ``not a career politician" and was ``outside [the] system." The responses about Sanders in Figure~\ref{fig:exemplartexts-bernie} convey a similar negative orientation toward political structures, but use somewhat different language. Here, respondents liked that Sanders was challenging ``the status quo" and fighting against the institutions that are often perceived as having undue leverage in Washington such as ``corporate america." It is interesting to note that these differing descriptions of Trump and Sanders as political outsiders align quite well with with left-wing versus right-wing populist typology identified by \textcite{lacatus2021populism}. Left-wing populists---a title often ascribed to Sanders---are known to take stances against corporations and the wealthy while right-wing populists---a label often attributed to Trump---espouse producerist and anti-political elite rhetoric. That these models detect these intricacies in language should be taken as an indication of the models' utility and validity. 


% Coef Plots for STM
\begin{figure}[t!]
\centering
   \begin{subfigure}[b]{0.45\textwidth}
	   \centering
	   \includegraphics[width=\textwidth]{Figures/STM-Coef-Outsider-Trump}
	   \caption{{\small Trump - STM Coefficient  Plot}}
	   \label{fig:coefplot-stm-trump} 
	\end{subfigure}
   \begin{subfigure}[b]{0.45\textwidth}
	   \centering
	   \includegraphics[width=\textwidth]{Figures/STM-Coef-Outsider-Bernie}
	   \caption{{\small Sanders - Coefficient  Plot}}
	   \label{fig:coefplot-stm-bernie} 
	\end{subfigure}
	\caption{Coefficient Plots - Effects of Electoral Inefficacy and Cynicism on Use of Political Outsider Topic}\label{fig:coefplot-stm}
	\vspace{-.25cm}
	    \vspace{-0.25cm}
    {\scriptsize \textit{Note:} 90\% confidence intervals shown. Estimates shown in \ref{fig:coefplot-stm-trump} and \ref{fig:coefplot-stm-bernie} taken from Tables~\ref{tab:outsider-trump} and \ref{tab:outsider-bernie}, respectively, in Appendix~\ref{app:stmresults}. Both models include extensive batteries of co-variates which have been omitted from this figure. }
\end{figure}

Were the politically alienated particularly inclined to view Trump and Sanders as political outsiders (Hypothesis~\ref{hyp:candevals-trump} and \ref{hyp:candevals-sanders})? I answer this question by regressing the proportion of a document dedicated to the outsider topic on measures of alienation (conducted separately for Trump and Sanders), while controlling for a range of factors that may influence support for either candidate. In the Trump model, the measures of alienation include electoral inefficacy and cynicism, and I control for such factors as anti-immigrant sentiments \parencite{sides2018identity,hooghe2018explaining}, attitudes toward Democratic-aligned social groups \parencite{mason2021activating,hopkins2021activation}, sexism \parencite{valentino2018mobilizing}, authoritarian tendencies \parencite{knuckey2020authoritarianism}, status threat \parencite{mutz2018status}, evangelical identification \parencite{margolis2020wants}, and partisanship. In the Sanders model, the measures of alienation include electoral inefficacy and political system illegitimacy, and I control for such factors as status threat \parencite{mutz2018status}, partisan strength, and sexism \parencite{sides2018identity}.\footnote{While \textcite{mutz2018status} is interested in the effect of status threat on Trump support (measured with the social dominance orientation index), I choose to include it in the Sanders model as it is possible that an affinity for outsiders is a reflection of a preference for social hierarchy, more broadly. As we will see, the results do not support this possibility.} Both models also include ideology, economic assessments, age, income, education, race, and sex as co-variates. I focus on the effects of the alienation measures on use of the outsider topic as shown in Figure~\ref{fig:coefplot-stm}, but the full model results can be found in Tables~\ref{tab:outsider-trump} and \ref{tab:outsider-bernie} of Appendix~\ref{app:stmresults}. 

The coefficient plot in Figure~\ref{fig:coefplot-stm-trump} reveals that both electoral inefficacy and cynicism have the expected positive effect on the use of the outsider topic to describe Trump.\footnote{In Tables~\ref{tab:outsider-trump} and \ref{tab:outsider-bernie} in Appendix~\ref{app:stmresults} I show that the use of other common topics is largely not driven by political alienation. Those who are more cynical, for instance, are no more likely to use any of the other top topics than those that are less cynical, and are actually less likely to use the `Make America Great Again' topic.} Given that both variables range between 0 and 1, the coefficients here (representing a one-unit change in the predictor) indicate that a move from the minimum to the maximum values of electoral inefficacy and cynicism produces a 1.4\% ($p<0.1$; one-tailed) and 4.2\% ($p<0.01$; one-tailed) increase, respectively, in the proportion of a document dedicated to the outsider topic. The size of these relationships are roughly the same as those found in previous applications of the Structural Topic Model to ANES open-ended responses \parencite[e.g.,][]{roberts2014structural}.\footnote{In Figure 18 on page 1080, \textcite{roberts2014structural} show that, among Republicans, an increase from 13 (high school) to 17 (college) years of education produces a $\approx$3\% reduction in the use of the `war' topic from the ANES Most Important Problem open-ended responses.} Additionally, because documents in mixed-membership models (such as the STM) are comprised of multiple topics, it is rare that a document will dedicate all of its content to a single topic. The document with the highest observed use of the `outsider' topic, for instance, only dedicated about 60\% of its content to that topic. Therefore, it is safe to say that a 1-4\% increase in the use of the `outsider' topic as a result of changes in feelings of electoral inefficacy or cynicism represents a meaningful effect, providing clear support for Hypothesis~\ref{hyp:candevals-trump}.  

In Figures~\ref{fig:coefplot-stm-bernie}, we see that political system illegitimacy---which taps both input- and output-focused alienation---has a similarly positive effect on the use of the outsider topic to describe Sanders. The coefficient estimate on political system illegitimacy suggests that a one-unit change in this measure is significantly related to a 13.3\% ($p<0.10$; one-tailed) increase in the proportion of the document dedicated to the outsider topic. The same cannot be said of electoral inefficacy, as the relationship between this measure and use of the outsider topic to describe Sanders appears negative but statistically insignificant. A possible reason for this null effect is the selection process for providing open-ended responses in the ISCAP panel. Only those that stated a preference for Sanders in the Democratic primary were asked to justify their preferences, whereas those that said they likely wouldn't vote in the primary were not asked for their justifications.\footnote{This issue does not apply to the ANES data, as all respondents where asked what (if anything) they liked about Trump, regardless of their partisanship or candidate preference in the primary elections.} Figure~\ref{fig:iscap-alienation-means} in Appendix~\ref{app:desc} reveals that individuals that stated they were unlikely to vote in the primary were far more electorally inefficacious compared to supporters of any other candidates, including Sanders supporters, so while these non-voters may have indicated an affinity for Sanders due to his outsider status if asked, the data do not allow me to investigate this possibility. These results with respect to political system illegitimacy provide suggestive, but not definitive, evidence in favor of Hypothesis~\ref{hyp:candevals-sanders}. 
	
My next task is to examine the effects of input-based and output-based alienation on voting behaviors in the 2016 presidential election. As noted above, I estimate separate multinomial logit models of vote choice for the primaries and for the general election. Interpretation of the coefficients from a multinomial logit is notoriously tricky, however, as the coefficients represent the change in the log-odds of selecting a particular outcome \textit{over some baseline category} as the result of a one-unit increase in the predictor. Instead of assessing statistical significance from the regression table, researchers are encouraged to calculate and interpret more substantively meaningful quantities of interest \parencite{king2000making,paolino2020predicted}. Therefore, for both models, I simulate the predicted probability of selecting each outcome (along with 90\% confidence intervals) as electoral inefficacy and cynicism move from their lowest (0) to their highest (1) values using the observed values approach \parencite{hanmer2013behind}.

% Vote Choice in Primary Election
\begin{figure}[!ht]
	\centering
	\includegraphics[width=0.8\linewidth]{Figures/Probs-Primary}
	\caption{Predicted Probabilities - Vote Choice in 2016 Primary Election}\label{fig:primary}
	\vspace{-8pt}
	{\scriptsize \textit{Note}: 90\% confidence intervals given. Estimates come from Multinomial Logit in Table~\ref{tab:primary} of Appendix~\ref{app:votemodels} \par}
\end{figure}

I begin by examining the effects of alienation in the open primary elections held on Super Tuesday in 2016. The model's output is presented in Table~\ref{tab:primary} of Appendix~\ref{app:votemodels} and the predicted probabilities of vote choice are presented in Figure~\ref{fig:primary}. Looking at the plots in the left column of Figure~\ref{fig:primary}, we see that electoral inefficacy had no meaningful effect on the Trump vote, and a negative but statistically insignificant effect on the Sanders vote (-0.043, 90\% CI[-0.102,0.019]). This indicates that output-based alienation---at least in this particular election---does not motivate people to vote for outsiders in the way that was anticipated by Hypothesis~\ref{hyp:input-vote}. If anything, output-based alienation appears to promote abstention as indicated by the positive (0.031, 90\% CI[-0.057,0.119]), but statistically insignificant, relationship between electoral inefficacy and the decision to not vote in the bottom-left plot of Figure~\ref{fig:primary}. These results are more supportive of Hypothesis~\ref{hyp:input-abstain}, but should not be considered conclusive. 

% cynicism is clearly related to voting for Sanders or Trump in the primaries: moving from the lowest (0) to the highest (2) value of cynicism significantly increases the probability of voting for Sanders 7.8\% (90\% CI [0.035,0.117]) and for Trump by 8.3\% (90\% CI [0.007,0.135]). Figure~\ref{fig:primary} also shows that cynicism was negatively related to abstention, as well as voting for some candidate than Trump or Sanders: moving across the range of the cynicism scale reduced the probability of not voting by 10.3\% (90\% CI[-0.192,-0.020]) and the probability of voting for some other candidate by 5.2\% [90\% CI[-0.156, 0.047]), though this effect is only significant in the former case. 

In the right column of Figure~\ref{fig:primary} we see that the effects of cynicism on vote choice in the primaries are more favorable to Trump and Sanders. Moving from the lowest to highest values of cynicism increase the probability of voting for Sanders by 7.0\% (90\% CI[0.018,0.117]) and for Trump by 1.7\% (90\% CI[-0.089,0.095]), though the latter effect does not quite reach statistical significance. Interestingly, we also see increases in cynicism reducing abstention by 3.2\% (90\% CI[-0.111,0.045]). Firm conclusions should not be drawn here as this effect does not reach statistical significance, but given that alienation was predictive of liking Sanders for his outsider qualities, and that output-alienation was related to the Sanders vote, these results are consistent with protest voting. I conclude that Hypothesis~\ref{hyp:output-voting} has mixed support with respect to Sanders in the primary elections. 

% Vote Choice in General Election
\begin{figure}[!ht]
	\centering
	\includegraphics[width=0.8\linewidth]{Figures/Probs-General.pdf}
	\caption{Predicted Probabilities - Vote Choice in 2016 General Election}\label{fig:general}
	\vspace{-8pt}
	{\scriptsize \textit{Note}: Estimates come from Multinomial Logit in Table~\ref{tab:general} in Appendix~\ref{app:votemodels} \par}
\end{figure}

Now I turn to examining the effects of input- and output-based alienation in the general election. Table~\ref{tab:general} of Appendix~\ref{app:votemodels} shows the output from the multinomial logit and Figure~\ref{fig:general} shows the predicted probability of voting for Clinton, Trump, some other candidate, or not voting at all as cynicism and electoral inefficacy vary from their minimum to their maximum values. In the left column of Figure~\ref{fig:general}, we see that electoral inefficacy did not encourage the Trump vote, demonstrating an unexpected, but insignificant, negative effect (-0.018, 90\% CI[-0.055,0.017]). While there is a slight positive relationship between electoral inefficacy and voting for a third-party, this relationship is similarly insignificant. There is essentially no effect of electoral inefficacy on the Clinton vote or abstention. These results appear to support neither Hypothesis~\ref{hyp:input-abstain} nor \ref{hyp:input-vote}---input-based alienation did not boost turnout for Trump, nor did it discourage participation on election day.

Turning now to the right column of Figure~\ref{fig:general}, it is clear that cynicism played a larger role than electoral inefficacy in determining vote choice. Moving from the lowest to highest values of cynicism significantly reduces the probability of abstention by 3.8\% (90\% CI[-0.077,-0.001]) while simultaneously increasing the probability of voting for Trump by 4.4\% (90\% CI[0.005,0.082]). This is precisely what we would expect if output-based alienation inspired protest votes in favor of Trump. Equally interesting is the fact that cynicism appears to reduce the probability of voting for the political insider in the race, Hillary Clinton (-0.032, 90\% CI[-0.065,-0.001]), and increase the probability of voting third-party (0.028, 90\% CI[0.002,0.051]). In total, it appears that cynicism played a key role in generating protest votes in Trump's favor---and to Clinton's detriment---consistent with Hypothesis~\ref{hyp:output-voting}.

%\footnote{The effects of cynicism on the Clinton and third-party vote is significant at $\alpha=0.10$ using a one-tailed ($100(1-2\alpha)=80\%$) confidence intervals \parencite[][174]{steiger2004beyond}.} 

% Turnout (1988-2016)
\begin{figure}[t!]
    \centering
    \includegraphics[width=0.6\textwidth]{Figures/Coef-Turnout-Cynicism.pdf}
    \caption{{\small `Cynicism' Coefficient}}    
    \label{fig:turnoutcoef-cynicism}
\end{figure}

A possible objection to my argument that Trump's outsider candidacy is responsible for uniquely boosting turnout in this election is that cynicism may be related to turnout even in elections with more traditional or establishment-type candidates. To alleviate this concern, I estimate models of voter turnout for each of the last eight elections and then compare the coefficients on cynicism with the results presented in Figure~\ref{fig:turnoutcoef-cynicism}. The only election year in which cynicism appears related to turnout is 2016, where the relationship appears positive and significant. The ability of the political outsider, Trump, to motivate the politically alienated to support him at the ballot box is a phenomenon that appears unique to 2016. 




%One might speculate that the Obama campaign's more positive message of ``hope and change" inspired the politically alienated to support him at the polls, but further analysis suggests that cynical voters supported Obama and McCain at similar rates in that election (see Table~\ref{tab:obamavote} of Appendix~\ref{app:votemodels}). 

%This finding is noteworthy when considered alongside the open-ended responses about Trump: the electorally inefficacious may have liked Trump for his outsider qualities, but their negative orientation towards elections precluded them from expressing these sentiments through the vote. 

%least to the most cynical attitudes towards government significantly increases the probability of voting for Trump by 7.6\% (90\% CI[0.036,0.115]) and significantly decreases the probability of not voting (i.e., increases turnout) by 5.3\% (90\% CI[-0.094,-0.019]). Cynicism also appears to reduce the probability of voting for Clinton by 3.7\% (90\% CI[-0.077,0.001]) and for some other (third-party) candidate by by 1.6\% (90\% CI[-0.008,0.040]), but the confidence intervals for these differences in predicted probabilities clearly contain zero. Thus it appears that cynicism motivated turnout in favor of the political outsider in the general election (Trump) just as it did in the primaries; these findings are clearly consistent with Hypothesis~\ref{hyp:cyn-voting}.

%Turning now to the right column of Figure~\ref{fig:general}, we see that the effects of election unresponsiveness are again far less pronounced than the effects of cynicism: moving across the range of the election unresponsiveness scale produces a slight decrease in the probability of voting for Clinton (1.6\%, 90\% CI[-0.051,0.017]) and a slight increase in the probability of not voting at all (1.4\%, 90\% CI[-0.016,0.046]), though neither of these differences are statistically significant. Election unresponsiveness also appears to have essentially no relationship with voting for Trump or some other (third-party) candidate. Here again I have found support for part of Hypothesis~\ref{hyp:elect-voting} that election unresponsiveness would be unrelated to one's decision to turnout, and consequently, unrelated to one's decision to vote for Trump in 2016. 

%In order to fully support the expectations of Hypothesis~\ref{hyp:elect-voting}, however, I must also show that election unresponsiveness is negatively related to turnout in typical national elections. To do this, I use a series of logistic regressions to model turnout in each of the last eight presidential elections (1988-2016). These models include cynicism and election unresponsiveness as the primary predictors along with a range of co-variates including partisan strength, eduction, income, age, and indicators for `independent' and `white.' I have not hypothesized about the effects of cynicism in typical national elections, but if my argument is correct that the politically alienated channel their frustrations with the political system through their support for political outsiders---and given that Trump is the only outsider to compete as a major party's candidate in a general election in recent memory---then we might expect cynicism to only be related to turnout in 2016. From these models, I plot the \textit{cynicism} and \textit{election unresponsiveness} logit coefficients in Figures~\ref{fig:turnoutcoef-cynicism} and \ref{fig:turnoutcoef-unresponsiveness}, respectively.\footnote{Full model results are given in Table~\ref{tab:turnout} of Appendix~\ref{app:votemodels}.} 
 
%While I have largely found support for my expectation regarding the effects of election unresponsiveness in the 2016 presidential election, my argument rests on the assumption that these effects are unique to this election cycle due to the presence of two `political outsider' candidates (i.e., Sanders and Trump). To determine if the effects of cynicism and election unresponsiveness that I have recovered are limited to the 2016 election---at least with regard to their ability to motivate one's decision to voice their feelings of alienation through their vote---I use a series of logistic regressions to model turnout in each of the last eight presidential elections (1988-2016). These models include cynicism and election unresponsiveness as the primary predictors along with a range of co-variates including partisan strength, eduction, income, age, and indicators for `independent' and `white.' Then, I plot the \textit{cynicism} and \textit{election unresponsiveness} logit coefficients from these models in Figures~\ref{fig:turnoutcoef-cynicism} and \ref{fig:turnoutcoef-unresponsiveness}, respectively.\footnote{Full model results are given in Table~\ref{tab:turnout} of Appendix~\ref{app:votemodels}.} 

% Turnout (1988-2016)
%\begin{figure*}[t!]
%    \centering
%    \begin{subfigure}[b]{0.475\textwidth}
%        \centering
%        \includegraphics[width=\textwidth]{Figures/Coef-Turnout-Cynicism.pdf}
%        \caption{{\small `Cynicism' Coefficient}}    
%        \label{fig:turnoutcoef-cynicism}
%    \end{subfigure}
%    \hfil
%    \begin{subfigure}[b]{0.475\textwidth}   
%        \centering 
%        \includegraphics[width=\textwidth]{Figures/Coef-Turnout-Unresponsive.pdf}
%        \caption{{\small `Election Unresponsiveness' Coefficient}}    
%        \label{fig:turnoutcoef-unresponsiveness}
%    \end{subfigure}
%    \caption{Coefficient Plots from Turnout Models, Estimated in Each Presidential Election Year, 1988-2016}\label{fig:turnoutcoefs}
%    \vspace{-8pt}
%    {\scriptsize \textit{Note:} Estimates from Table~\ref{tab:turnout} in Appendix~\ref{app:votemodels} \par}
%\end{figure*}

%The coefficient estimates for cynicism shown in Figure~\ref{fig:turnoutcoef-cynicism} suggest that it is rarely a motivator for turnout in general elections: in six of the last eight elections, the \textit{cynicism} coefficient is statistically indistinguishable from zero. This is not the case in 2008 and 2016, however, as cynicism is positive and significantly related to turnout in these elections. At the same time, election unresponsiveness appears to consistently depress turnout, with the coefficient on election unresponsiveness being negative and statistically distinguishable from zero in every presidential election between 1988 and 2012 (Figure~\ref{fig:turnoutcoef-unresponsiveness}). It is only during the 2016 general election that election unresponsiveness has no discernible effect on turnout, as the 90\% confidence interval on the estimate overlaps with zero.\footnote{The 90\% confidence interval's overlap with zero suggests that this estimate is not significant using a one-tailed test at $\alpha=0.05$. However, this effect does become significant if one applies a one-tailed test at $\alpha=0.10$.} This finding provides further support to Hypothesis~\ref{hyp:elect-voting} that election unresponsiveness would be negatively related to turnout in typical national elections. 








%%%%%%%%%%%%%%%%%%%%%%%%%%%%%%%%%%%%%%%%%%%%%%%%%%%%%%%%%%%%%%%%%%%%%%%%%%%%%
%%%%%%%%%%%%%%%%%%%%%%%%%%%%%%%%%%%%%%%%%%%%%%%%%%%%%%%%%%%%%%%%%%%%%%%%%%%%%
%%                                                                         %%
%%   %%%%%  %%%%%  %    %  %%%%%  %      %    %  %%%%%  %  %%%%%  %    %   %%
%%   %      %   %  %%   %  %      %      %    %  %      %  %   %  %%   %   %%  
%%   %      %   %  % %  %  %      %      %    %  %%%%%  %  %   %  %  % %   %%
%%   %      %   %  %   %%  %      %      %    %      %  %  %   %  %   %%   %%
%%   %%%%%  %%%%%  %    %  %%%%%  %%%%%  %%%%%%  %%%%%  %  %%%%%  %    %   %%
%%                                                     					   %%
%%%%%%%%%%%%%%%%%%%%%%%%%%%%%%%%%%%%%%%%%%%%%%%%%%%%%%%%%%%%%%%%%%%%%%%%%%%%%
%%%%%%%%%%%%%%%%%%%%%%%%%%%%%%%%%%%%%%%%%%%%%%%%%%%%%%%%%%%%%%%%%%%%%%%%%%%%%

\section{Conclusion}\label{sec:conclusion}
Political alienation describes a feeling of estrangement from the inputs and outputs to the political system, and is often accompanied by a sense of negativism towards political processes and structures \parencite{horton1962powerlessness}. For those with input-based alienation, political participation is not seen as a useful mechanism for signaling discontent, and for those with output-based alienation, opportunities to translate one's negative attitudes into action are rarely available. However, I have argued and demonstrated that the 2016 U.S. presidential election cycle was an opportunity for the politically alienated to cast protest votes at the national level due to the presence of two outsider candidates: Donald Trump and Bernie Sanders. I showed that the politically alienated---be it input- or output-based---were more likely to say they liked Trump and Sanders specifically because they stood opposed to the political system. Then, I showed that protest votes largely occurred through output-based alienation, increasing the likelihood of voting for Sanders in the primary and Trump in the general election, while at the same time increasing turnout (more so in the general election). It is important to note the the relationships I uncovered between alienation and both candidate evaluations and vote choice are robust to the inclusion of multiple factors that are known to influence Trump or Sanders support, such as status threat, social identities, racial resentment, modern sexism, and authoritarianism \parencite{mutz2018status,sides2018identity,hooghe2018explaining,hopkins2021activation,valentino2018mobilizing,knuckey2020authoritarianism}. Finally, I showed that the effect of output-based alienation on turnout in 2016 was unique to that election cycle---in the last eight general election, cynicism appears to have boosted turn only in 2016. 

This analysis has contributed to our understanding of the relationship between alienation and voting behavior in two clear ways. First, I have unpacked a mechanism underlying the protest vote in U.S. presidential elections. While \textcite{southwell1998electoral} suggested that output-based alienation promoted protest votes in favor of Perot in the 1992 election, the motivations underlying these votes had yet to be uncovered. My examination of the open-ended responses about Trump and Sanders reveals that their status as political outsiders was an important consideration for the politically alienated, and at least for those with output-based alienation, those feelings of alienation translated into a vote. Second, I have helped to round-out our understanding of how two non-traditional candidates could achieve such success in a presidential election. The role of identity in supporting the rise of Trump and Sanders is well established \parencite[e.g.,][]{sides2018identity,mason2021activating}, but my results suggest that a fuller understanding of their success requires consideration of the role played by anti-establishment sentiment. Even when the role of identity was taken into account, political alienation still drove candidate evaluations and vote choice in various ways. 

As a final consideration, it is important to note that I have remained agnostic about the specific sources of peoples' feelings of political alienation. As I have defined it, true feelings of alienation should have pre-existed the emergence of Trump and Sanders, and should not have developed due solely to the rhetoric of these two political outsiders. Indeed, Figure~\ref{fig:mean-alienation} shows that the ANES measure of electoral inefficacy increased only slightly between 2012 and 2012, while cynicism remained consistent during this same time period (with Democrats even showing a slight decrease). Given that our political system typically changes at a rather glacial pace, it makes sense that feelings of alienation would be slow-developing, as well. This is not to say, however, that feelings of political alienation that lie dormant cannot become politically relevant. In fact, I believe that this was likely the case in 2016---Trump and Sanders both used their platforms to help voters make the connection between their candidacies and voters' feelings of political alienation, which some have appeared to use motivation for their vote choice. Identifying the specific source of such attitudes remain beyond the scope of this project, but provide fertile ground for future research. 

%An important implication of my findings to consider is that there are apparent limits on candidates' ability to generate electoral support through criticism of the political system. Input-based alienation, for instance, did not appear to increase the likelihood of voting for either Trump or Sanders in my analysis. If anything, there was some suggestive evidence from the primaries that input-based alienation reduced turnout. It is not hard to imagine, then, how a candidate's attacks on the political system could unexpectedly backfire. Consider the 2021 Georgia Senate run-off election: following his loss to Joe Biden, Donald Trump repeatedly questioned the legitimacy of 2020 U.S. presidential election, including the results from the (largely GOP-controlled) state of Georgia. Initial accounts suggest that these attacks on the electoral process contributed to a decline in GOP voter turnout in the Senate run-offs \parencite{niesse2021}, tipping the election to the two Democratic candidates. It would appear that guiding the politically alienated to channel their frustrations for the political system through their vote is a delicate and dangerous needle to thread. 


%My finding that alienation can be channeled through support for `political outsiders' is consequential given that typical indicators of alienation, such as the level of trust in government, have fallen rapidly in recent decades \parencite{Citrin2018}. If feelings of alienation continue to swell in the American public, more opportunistic `outsiders' may emerge to try and ride the wave of anti-establishment sentiment. My results also help us to understand that the way in which feelings of alienation will be expressed depends on whether they are feelings of normlessness or powerlessness: both dimensions of alienation can change peoples \textit{attitudes} about the political world, but have different effects on how we \textit{behave} in it. Understanding this distinction is crucial to determining exactly how feelings of alienation will manifest in the American public. 

%An important contribution of this work is that it helps to round-out our understanding of how a candidate like Trump, with no prior office-holding experience or political skills, could become the nominee for a major American political party and eventually be elected president. Previous explanations have largely pointed to the power of identity as key drivers of Trump's success \parencite{sides2018identity} and, to be fair, identity was clearly a key component of the 2016 election. However, my argument regarding the effects of alienation is purely attitudinal as I have no reason to believe that the alienated share a cohesive sense of identity, especially considering that these attitudes are held among members of opposing parties.

%The lack of a cohesive identity is also what distinguishes alienation from another label that was commonly used to describe the appeal of both Trump and Sanders: populism. Populism is typically characterized as a struggle between the people vs. the elites or us vs. them \parencite{Lee2019}. As I have defined it, alienation certainly entails a disdain for ruling elites, but does not necessarily require one to identify with `the people.' In fact, the \textit{isolation} dimension of alienation may even entail an outright rejection of the will of the people. Populism and alienation may share similar features (i.e. resentment for the political system), but they are not necessarily one and the same. 

% entire impact of the growth of political alienation in the American public requires that we u
%feelings of normlessness, for instance, were shown to prime participation in the voting process. This could potentially be taken as a sign of a healthy and functioning democracy since those who are most dissatisfied with the current ``outputs" \parencite{easton1965systems} of our political system are using the proper legal channels to make their concerns known. \textit{Powerlessness}, on the other hand, typically leads people to forego the political process
%may, in fact, be part of a healthy and functioning democracy, as they are an indicator of dissatisfaction with the current `outputs" \parencite{easton1965systems} of government which can potentially be remedied through more traditional channels of political action (e.g., elections, ballot initiatives). If alienation in the American public begins to manifest as feelings of \textit{powerlessness}, however, those traditional channels of political action may be ruled out in favor of forms of political action that are far more damaging to our democracy. The extent to which feelings of \textit{powerlessness} have developed as a result of the 
%Both Donald Trump and Bernie Sanders successfully branded themselves as `political outsiders,' thereby attracting the support of politically alienated individuals of all partisan stripes that had become disenchanted with the American political system. 
%The findings presented here also call for scholars to reconsider the qualities that citizens find attractive in their candidates. In the past, it may have been reasonable to assume that voters want candidates with relevant political experience and connections to the parties, yet many alienated voters liked Trump and Sanders specifically because they lacked these features. In an era where confidence in government is lacking \parencite{Citrin2018} and the parties have become increasingly polarized, we need to ask: what is the value of office-holding experience if voters feel that politicians are untrustworthy, corrupt, and do not represent their interests?

























%%%%%%%%%%%%%%%%%%%%%%%%%%%%%%%%%
     %%%% BIBLIOGRAPHY %%%%
%%%%%%%%%%%%%%%%%%%%%%%%%%%%%%%%%
\clearpage
\printbibliography


%%%%%%%%%%%%%%%%%%%%%%%%%%%%%%%%%
       %%%% APPENDIX %%%%
%%%%%%%%%%%%%%%%%%%%%%%%%%%%%%%%%
% TITLE PAGE
\clearpage
\appendix
\begin{titlepage}
   \vspace*{\stretch{1.0}}
   \begin{center}
      \Large\textbf{Alie(n)ation: Political Outsiders in the 2016 U.S. Presidential Election}\\
      \large Appendix \\
      \large\textit{Maxwell B. Allamong}
   \end{center}
   \vspace*{\stretch{2.0}}
\end{titlepage}



\begin{appendices}
\begin{refsection}


% ISCAP
\section{ISCAP Data}
\subsection{Political System Legitimacy}\label{app:legitimacy}
The first principal component in these items explains 46\% of the variance in the outcome. 

\begin{table}[ht!]
\caption{Political System Legitimacy - PCA Loadings}
\begin{center}
\label{table:principalcomponent}
\begin{tabular}{ll}
\hline
Survey Item                                & PC1 Loading \\ \hline
Would rather live under our system of gov. than any other & 0.510\\
System of gov. needs serious changes & 0.443\\
Our form of gov. is best for representing citizen's interests & 0.537\\
Feel very critical of our political system & 0.505\\
\hline
\end{tabular}
\end{center}
\end{table}


\subsection{Descriptive Statistics}\label{app:desc}
\begin{figure}[h!]
	\centering
	\includegraphics[width=\linewidth]{Figures/means-iscap-alienation}	
	\caption{Mean of ISCAP Alienation Measures by Candidate Preference in Democratic Primary}\label{fig:iscap-alienation-means}
	\vspace{-0.25cm}
	{\scriptsize \textit{Note:} Sample limited to Democrats only as Republicans were not asked about their candidate preference in the Democratic primary}
\end{figure}

\subsection{Co-variates}
\singlespacing
Original coding schemes provided below. All variables were rescaled to range between 0 and 1 for all analyses. Question wording is provided were necessary.
\begin{itemize}
	\item Age
		\begin{itemize}
			\item Coding: in years
		\end{itemize}
		
	\item Black - 
		\begin{itemize}
			\item Coding: (1) Yes, (0) No
		\end{itemize}
				
	\item Economic Assessments - Retrospective/Personal
		\begin{itemize}
			\item Question Wording: ``We are interested in how people are getting along financially these days. Would you say that you and your family living here are better off, worse off, or just about the same financially as you were a year ago?
			\item Coding: (1) A lot better off, (2) A little better off, (3) A little worse off, (4) A lot worse off, (5) Just about the same
			\item These items were re-arranged so that the highest value represented beliefs that the one's financial situation had gotten a lot better, the lowest value represented beliefs that one's financial situation had gotten a lot worse, and middling values represented beliefs that one's financial situation had only gotten a little better, stayed the same, or gotten a little worse. 
		\end{itemize}
		
	\item Economic Assessments - Retrospective/Sociotropic
		\begin{itemize}
			\item Question Wording: ``Thinking about the economy in the country as a whole, would you say that over the past year the nation's economy has gotten better, stayed about the same, or gotten worse?"
			\item Coding: (1) Gotten a lot better, (2) Gotten a little better, (3) Gotten a little worse, (4) Gotten a lot worse, (5) Stayed about the same
			\item These items were re-arranged so that the highest value represented beliefs that the economy as a whole had gotten a lot better, the lowest value represented beliefs that the economy had gotten a lot worse, and middling values represented beliefs that the economy had only gotten a little better, stayed the same, or gotten a little worse. 
		\end{itemize}
		
	\item Education
		\begin{itemize}
			\item Coding: (1) Less than high school, (2) High school diploma, (3) Some college, no bachelors degree, (4) Bachelors or above
		\end{itemize}
		
	\item Female
		\begin{itemize}
			\item Coding: (1) Yes, (0) No
		\end{itemize}
		
	\item Income
		\begin{itemize}
			\item Coding: in quintiles
		\end{itemize}
		
	\item Ideology 
		\begin{itemize}
			\item Coding: (1) Extremely liberal, (2) Liberal, (3) Slightly liberal, (4) Moderate, middle of the road, (5) Slightly conservative, (6) Conservative, (7) Extremely conservative 
		\end{itemize}
		
	\item Modern Sexism Index (MSI)
		\begin{itemize}
			\item Item comes from Wave 11 (Sep-Nov 2016) of ISCAP Panel
			\item Created by additively indexing responses to the three statements below. Each item re-coded so that higher values = more sexism before the items are combined (more sexist answer in parentheses). 
			\begin{itemize}
				\item Item 1: ``When women demand equality these days, they are actually seeking special favors." (Agree)
				\item Item 2: ``Women often miss out on good jobs because of discrimination." (Disagree)
				\item Item 3: ``Women who complain about harassment cause more problems than they solve." (Agree)
				\item Coding: (1) Agree strongly, (2) Agree somewhat, (3) Neither agree nor disagree (4) Disagree somewhat, (5) Disagree strongly
			\end{itemize}
		\end{itemize}
		
	\item Partisan Strength 
		\begin{itemize}
			\item Coding: (1) Independent, (2) Leaning partisan, (3) Partisan, (4) Strong partisan
		\end{itemize}
				
	\item Social Dominance Orientation (SDO)
		\begin{itemize}
			\item Created by averaging responses to the four statements below. Each item re-coded so that higher values = strong social dominance orientation before the four are combined (SDO responses in parentheses). 
			\begin{itemize}
			\item Item 1: ``In setting priorities, we must consider all groups." (Oppose)
				\item Item 2: ``We should not push for group equality." (Favor)
				\item Item 3: ``Group equality should be our ideal." (Oppose)
				\item Item 4: ``Superior groups should dominate inferior groups." (Favor)
				\item Coding: (1) Extremely oppose to (10) Extremely favor
			\end{itemize}
		\end{itemize}
		
	\item White 
		\begin{itemize}
			\item Coding: (1) Yes, (0) No
		\end{itemize}
\end{itemize}


% Histogram
%\begin{figure}[h!]
%	\centering
%	\includegraphics[width=0.9\linewidth]{Figures/histograms.pdf}
%	\caption{Histograms of Political Alienation Scale, 1988-2016}\label{app:hist}
%\end{figure}

% Mean of alienation by PID
%\begin{figure}[hp!]
%	\centering
%	\includegraphics[width=0.9\linewidth]{Figures/alienation-means.pdf}
%	\caption{Mean of Political Alienation Index by Partisanship, 1988-2016}\label{app:alienation-means}
%\end{figure}
%\clearpage


% Factor Loadings
%\begin{table}[hp!]
%	\centering
%	\caption{Factor Loading - Political Alienation Index (2016)}\label{tab:factanal}
%	\begin{tabular}{l|c}\hline
%		\textit{Trust} & 0.77\\
%		\textit{Corruption} & 0.67\\
%		\textit{Few or All} & 0.72\\
%		\textit{Elections} & 0.53 \\ \hline
%	\end{tabular}
%\end{table}

%Principal Components Analysis
%\begin{figure}[htp!]
%	\centering
%	\includegraphics[width = 0.7\linewidth]{Figures/alienation-pca.pdf}
%	\caption{PCA of Political Alienation Index, 2016}\label{fig:pca}
%\end{figure}











%%%% ANES CODING %%%%
\clearpage
\section{ANES Data}\label{app:variablecoding}

\subsection{Descriptive Statistics}
% Means of Alienation Over Time, by PID
\begin{figure}[h!]
\centering
   \begin{subfigure}[b]{0.75\textwidth}
	   \centering
	   \includegraphics[width=\linewidth]{Figures/Mean-ANES-ElectInefficacy}
	   \caption{Electoral Inefficacy}
	   \label{fig:mean-electineff} 
	\end{subfigure}
\\ 
	\begin{subfigure}[b]{0.75\textwidth}
		\centering
	   \includegraphics[width=\linewidth]{Figures/Mean-ANES-Cynicism} 
	   \caption{Cynicism}
	   \label{fig:mean-cyn}
	\end{subfigure}
	\caption{Mean of ANES Measures of Electoral Inefficacy and Cynicism, 1988-2016}\label{fig:mean-alienation}
	%\vspace{-.25cm}
	%{\scriptsize \textit{Note:} Researcher designated labels given in parentheses, FREX words are those that are both frequent and exclusive to a topic}
\end{figure}



\subsection{Co-variates}
\singlespacing
Original coding schemes provided below. All variables were rescaled to range between 0 and 1 for all analyses. Question wording is provided were necessary.

\begin{itemize}
	\item Age
		\begin{itemize}
			\item Coding: in years
		\end{itemize}
		
	\item Anti-Immigrant Attitudes
		\begin{itemize}
			\item Created by averaging responses to the three question below. Each item re-coded so that higher values = more anti-immigrant sentiment before the three are combined
			\item Birthright citizenship
			\begin{itemize}
				\item Question Wording: ``Some people have proposed that the U.S. Constitution should be changed so that the children of unauthorized immigrants do not automatically get citizenship if they are born in this country. Do you favor, oppose, or neither favor nor oppose this proposal?"
				\item If R favors or opposes this change, strength of attitude is probed
				\item Coding: (1) Favor a great deal, (2) Favor a moderate amount, (3) Favor a little, (4) Neither favor nor oppose, (5) Oppose a little, (6) Oppose a moderate amount, (7) Oppose a great deal
			\end{itemize}
			\item Childhood arrivals
			\begin{itemize}
				\item Question Wording: ``What should happen to immigrants who were brought to the U.S. illegally as children and have lived here for at least 10 years and graduated high school here? Should they be sent back where they came from, or should they be allowed to live and work in the United States?"
				\item Upon answering the above prompt, strength of attitude if probed
				\item Coding: (1) Should send back - favor a great deal, (2) Should send back - favor a moderate amount, (3) Should send back - favor a little, (4) Should allow to stay - favor a little, (5) Should allow to stay - favor a moderate amount, (6) Should allow to stay - favor a great deal
			\end{itemize}
			\item Build wall with Mexico
			\begin{itemize}
				\item Question Wording: ``Do you favor, oppose, or neither favor nor oppose building a wall on the U.S. border with Mexico?
				\item If R favors or opposes this change, strength of attitude is probed
				\item Coding: (1) Favor a great deal, (2) Favor a moderate amount, (3) Favor a little, (4) Neither favor nor oppose, (5) Oppose a little, (6) Oppose a moderate amount, (7) Oppose a great deal
			\end{itemize}
		\end{itemize}
		
	\item Anti-Trade Attitudes
		\begin{itemize}
			\item Question Wording: ``Do you favor, oppose, or neither favor nor oppose the U.S. making free trade agreements with other countries?"
			\item If R favors or opposes, strength of attitude is probed
			\item Coding: (1) Favor a great deal, (2) Favor moderately, (3) Favor a little, (4) Neither favor nor oppose, (5) Oppose a little, (6) Oppose moderately, (7) Oppose a great deal
		\end{itemize}
		
	\item Child-Rearing Authoritarianism
		\begin{itemize}
			\item Created by averaging responses to the following four statements. Authoritarian traits are indicated in italics.
			\item Question Wording: ``Please tell me which one you think is more important for a child to have\ldots"
			\begin{itemize}
				\item Item 1: Independence or \textit{Respect for elders}
				\item Item 2: Curiosity or \textit{Good manners}
				\item Item 3: \textit{Obedience} or Self-reliance
				\item Item 4: Being considerate or \textit{Well-behaved}
			\end{itemize}
			\item Coding: (1) Non-authoritarian trait, (2) Both, (3) Authoritarian trait
		\end{itemize}
		
	\item China as Threat 
		\begin{itemize}
			\item Re-coded so that higher values = strong beliefs that China's military is a threat
			\item Question Wording: ``Do you think China's military is a major threat to the security of the United States, a minor threat, or not a threat?
			\item Coding: (1) Major threat, (2) Minor threat, (3) Not a threat
		\end{itemize}
		
	\item Democratic-Aligned Group Thermometers
		\begin{itemize}
			\item Created by averaging responses to the feeling thermometers for the following four Democratic-aligned groups: Blacks, Muslims, LGBT, and Hispanics
			\item Coding: (0) Least favorable attitudes, (100) Most favorable attitudes
		\end{itemize}
		
	\item Economic Assessments - Prospective/Sociotropic
		\begin{itemize}
			\item Question Wording: ``What about the next 12 months? Do you expect the economy in the country as a whole to get better, stay about the same, or get worse? [If R answers `get better' or `get worse'], Much better or somewhat better?/Much worse or somewhat worse?"
			\item Coding: (1) Get much better, (2) Get somewhat better, (3) About the same, (4) Get somewhat worse, (5) Get much worse
			\item These items were re-arranged so that the highest value represented beliefs that the economy had gotten a lot better, the lowest value represented beliefs that the economy had gotten a lot worse, and middling values represented beliefs that the economy had only gotten somewhat better, stayed the same, or somewhat worse. 
		\end{itemize}	
						
	\item Economic Assessments - Retrospective/Sociotropic
		\begin{itemize}
			\item Question Wording: ``Now thinking about the economy in the country as a whole, would you say that over the past year the nation's economy has gotten better, stayed about the same, or gotten worse? [If R answers `gotten better' or `gotten worse'], Much better or somewhat better?/Much worse or somewhat worse?"
			\item Coding: (1) Much better, (2) Somewhat better, (3) About the same, (4) Somewhat worse, (5) Much worse
			\item These items were re-arranged so that the highest value represented beliefs that the economy had gotten a lot better, the lowest value represented beliefs that the economy had gotten a lot worse, and middling values represented beliefs that the economy had only gotten somewhat better, stayed the same, or somewhat worse. 
		\end{itemize}
		
	\item Education
		\begin{itemize}
			\item Coding: (1) Less than high school, (2) High school diploma, (3) Some college, no bachelors degree, (4) Bachelors or above
		\end{itemize}
		
	\item Evangelical
		\begin{itemize}
			\item Coding: (1) Yes, (0) No
		\end{itemize}
		
	\item Female
		\begin{itemize}
			\item Coding: (1) Yes, (0) No
		\end{itemize}
		
	\item Ideology 
		\begin{itemize}
			\item Coding: (1) Extremely liberal, (2) Liberal, (3) Slightly liberal, (4) Moderate, middle of the road, (5) Slightly conservative, (6) Conservative, (7) Extremely conservative 
		\end{itemize}
		
	\item Income
		\begin{itemize}
			\item Coding: in quintiles
		\end{itemize}
		
	\item Independent  
		\begin{itemize}
			\item Pure independents only
			\item Coding: (1) Yes, (0) No
		\end{itemize}
		
	\item Modern Sexism Index (MSI)
		\begin{itemize}
			\item Created by additively indexing responses to the three statements below. Each item re-coded so that higher values = more sexism before the items are combined (more sexist answer in parentheses). 
			\begin{itemize}
				\item Item 1: ``When women demand equality these days, How often are they are actually seeking special favors?" (Always)
				\begin{itemize}
					\item Coding: (1) Always, (2) Most of the time, (3) About half the time, (4) Some of the time, (5) Never
				\end{itemize}
				\item Item 2: ``Should the news media pay more attention to discrimination against women, less attention, or the same amount of attention they have been paying lately? [If R answers `more attention' or `less attention'], how much more/less attention should media pay to discrimination against women?
				\begin{itemize}
					\item Coding: (1) A great deal more attention, (2) Somewhat more attention, (3) A little more attention, (4) Same amount of attention, (5) A little less attention, (6) Somewhat less attention, (7) A great deal less attention
				\end{itemize}
				\item Item 3: ``When women complain about harassment, how often do they cause more problems than they solve?" (Always)
				\begin{itemize}
					\item Coding: (1) Always, (2) Most of the time, (3) About half the time, (4) Some of the time, (5) Never
				\end{itemize}
			\end{itemize}
		\end{itemize}
		
	\item Political Interest
		\begin{itemize}
			\item Coding: (1) Not much interested, (2) Somewhat interested, (3) Very much interested
		\end{itemize}
		
	\item Republican  
		\begin{itemize}
			\item Leaners included
			\item Coding: (1) Yes, (0) No
		\end{itemize}
		
	\item White 
		\begin{itemize}
			\item Coding: (1) Yes, (0) No
		\end{itemize}
\end{itemize}
%\begin{itemize}
%	
%	% Political Alienation
%	\item \textbf{Cynicism}
%	\begin{itemize}
%		\item `No Trust'
%		\begin{itemize}
%			\item Variables: V161215, trustgov$\_$trustgrev, VCF0604
%			\item Question Wording: ``How often can you trust the federal government in Washington to do what is right?"
%			\item Coding: 1 = Always, 2 = Most of the time, 3 = About half of the time, 4 = \textit{Some of the time}, 5 = \textit{Never}
%		\end{itemize}
%		\item `Big Interests'	
%		\begin{itemize}
%			\item Variables: V161216, trustgov$\_$bigintrst, VCF0605
%			\item Question Wording: ``Would you say the government is pretty much run by a few big interests looking out for themselves or that it is run for the benefit of all the people?"
%			\item Coding: 1 = \textit{Run by a few big interests}, 0 = For the benefit of all the people
%		\end{itemize}
%		\item Note: I create my measure of cynicism by giving a point for each alienated response (in italics) on the `No Trust' and `Big Interests' items, thus producing a scale that ranges from 0 (no alienated responses) to 2 (all alienated responses). 
%	\end{itemize}
%	
%	% Election Unresponsiveness
%	\item \textbf{Election Unresponsiveness}
%	\begin{itemize}
%		\item Variables: V161220, respons$\_$elections, VCF0624
%		\item Question Wording: ``How much do you feel that having elections makes the government pay attention to what the people think?"
%		\item Coding: 1 = A good deal, 2 = Some, 3 = Not much
%	\end{itemize}
%	
%	% Partisanship
%	\item \textbf{Partisanship}
%	\begin{itemize}
%		\item Variables: V161158x, pid$\_$self, VCF0301
%		\item Coding: indicators (1 = True, 0 otherwise) created for `Democrat,' `Independent,' and `Republican'
%		\item Note: `Democrat' is the reference category for several models included in this paper, therefore, only the `Republican' and `Independent' indicators are seen in the tables/figures the present the results of these models
%	\end{itemize}
%	
%	% Ideology
%	\item \textbf{Ideology}
%	\begin{itemize}
%		\item Variables: V161126, libcpre$\_$self, VCF0803
%		\item Coding: 1 = Extremely liberal, 2 = Liberal, 3 = Slightly liberal, 4 = Moderate, middle of the road, 5 = Slightly conservative, 6 = Conservative, 7 = Extremely conservative
%	\end{itemize}
%	
%	% White
%	\item \textbf{White}
%	\begin{itemize}
%		\item Variables: V161310x, dem$\_$racecps$\_$white, VCF0105a
%		\item Coding: 1 = White, 0 otherwise
%	\end{itemize}
%	
%	% Income
%	\item \textbf{Income}
%	\begin{itemize}
%		\item Variables: V161361x, inc$\_$incgroup$\_$pre, VCF0114
%		\item Coding: given in quintiles (1-5 continuous)
%	\end{itemize} 
%	
%	% Female
%	\item \textbf{Female}
%	\begin{itemize}
%		\item Variables: V161342, gender$\_$respondent$\_$x, VCF0104
%		\item Coding: 1 = Female, 0 otherwise
%	\end{itemize}
%	
%	% Age
%	\item \textbf{Age}
%	\begin{itemize}
%		\item Variables: V161267, dem$\_$age$\_$r$\_$x, VCF0101
%		\item Coding: 17-99 years of age, continuous
%	\end{itemize}
%	
%	% Education
%	\item \textbf{Education}
%	\begin{itemize}
%		\item Variables: V161270, dem$\_$edugroup$\_$x, VCF0110
%		\item Coding: 1 = Less than high school diploma, 2 = High school diploma or equivalent, 3 = Some college but no Bachelors degree, 4 = Bachelors degree or above
%	\end{itemize}
%	
%	% Political interest
%		\item \textbf{Political Interest}
%	\begin{itemize}
%		\item Variables: V161004, interest$\_$following, VCF0310
%		\item Coding: 1 = Not much interested, 2 = Somewhat interested, 3 = Very much interested
%	\end{itemize}
%	
%\end{itemize}		



\doublespacing


\clearpage
%%%% STRUCTURAL TOPIC MODEL %%%%
\section{Structural Topic Model}\label{app:stm}

%% PRE-PROCESSING
\subsection{Pre-Processing}\label{app:preprocessing}
Before estimating Structural Topic Models on the open-ended responses about Trump and Sanders, I started by pre-processing the texts which includes removing unnecessary punctuation, numbers, and stop words (e.g., ``it," ``what," ``is"), converting all characters to lowercase, and correcting spelling. I also chose to remove terms that appear in no more than one document. Following these pre-processing steps, I am left with 1,099 documents and 549 terms in the corpus of texts about Trump, and 174 documents and 133 terms in the corpus of texts about Sanders. 

%% MODEL SELECTION
\subsection{Model Selection}\label{app:modelselection}

% Selecting K
\begin{figure}[bp!]
\centering
   \begin{subfigure}[b]{0.6\textwidth}
	   \centering
	   \includegraphics[width=\textwidth]{Figures/selectingK-trump.pdf}
	   %\vspace{-.25cm}
	   \caption{Trump}
	   \label{fig:selectingK-trump} 
	\end{subfigure}
\\
	\begin{subfigure}[b]{0.6\textwidth}
		\centering
	   \includegraphics[width=\textwidth]{Figures/selectingK-bernie.pdf}
	   \caption{Sanders}
	   \label{fig:selectingK-sanders}
	\end{subfigure}
	\caption{Determining the Number of Topics to Model, Diagnostics}\label{fig:selectingK}
	%\vspace{-.25cm}
	%{\footnotesize \textit{Note:}}
\end{figure}

% Select Model Diagnostics
\begin{figure}[bp!]
\centering
   \begin{subfigure}[b]{0.6\textwidth}
	   \centering
	   \includegraphics[width=\textwidth]{Figures/selectModel-trump.pdf}
	   %\vspace{-.25cm}
	   \caption{Trump}
	   \label{fig:selectModel-trump} 
	\end{subfigure}
\\
	\begin{subfigure}[b]{0.6\textwidth}
		\centering
	   \includegraphics[width=\textwidth]{Figures/selectModel-bernie.pdf}
	   \caption{Sanders}
	   \label{fig:selectModel-sanders}
	\end{subfigure}
	\caption{Comparing Semantic Coherence and Exclusivity of Models with Various Initializations}\label{fig:selectModel}
	%\vspace{-.25cm}
	%{\footnotesize \textit{Note:}}
\end{figure}

\textcite{roberts2014structural} note that there is not necessarily a correct number of topics for any given corpus, so they recommend that researchers make this selection based on substantive knowledge that they many have about the content of the texts, and that they consider the purpose for which the texts will be used. Additionally, \textcite{roberts2019stm} provide the \texttt{searchK} function in their \texttt{stm} package to allow researchers a more empirically-driven method of selecting of the number of topics. Following this advice, I note that the 2008 ANES Likes/Dislikes about Candidates were manually coded by ANES staff into roughly 30 topics, so I expect roughly the same number of topics to be found in the 2016 responses about Trump. Unfortunately, I cannot rely on previous iterations of the ISCAP panel to guide me on the number of topics in the Sanders texts in a similar way. However, given that ISCAP respondents were asked to provide a one-sentence justification for preferring a particular primary candidate---whereas ANES respondents can provide up to 5 mentions---I suspect that the number of topics in the Sanders texts will be no more than, and perhaps less than, the number of topics in the Trump texts. With this in mind, I then proceed by using the \texttt{searchK} function to generate models that range in the number of topics---for the Trump texts I generate models ranging from 20 to 36 topics, and for the Sanders texts, I generate models that range from 5 to 19 topics. I generate performance diagnostics from these models such as held-out likelihood, residuals, semantic coherence, and lower bound and plot them in Figures~\ref{fig:selectingK-trump} and \ref{fig:selectingK-sanders}.

In selecting the number of topics, we are looking for the held-out likelihood and semantic coherence to be high while the residuals should be low. For the Trump texts, models with $\approx$27 topics seem to fit this pattern quite well, while $\approx$13 topics seems more appropriate for the Sanders text. After estimating models in this more narrow range, I ultimately settle on a model with 27 topics for the Trump texts and 13 topics for the Sanders texts. Because the results of the STM are sensitive to initialization, the last step before finalizing the model is to use the \texttt{selectModel} function to generate several models on either set of texts. From each of the model runs, I plot the semantic coherence and exclusivity, shown in Figure~\ref{fig:selectModel}. Notice that models 1 through 6 all show roughly the same values of semantic coherence and exclusivity. Because the models performed so similarly, I manually inspected the topic content from several of the models, and selected the model where the FREX (Frequent-Exclusive) words logically went together and a common theme could be discerned from exemplar texts. 



\clearpage
%% TOPIC MODEL RESULTS
\subsection{Model Results}\label{app:stmresults}

% All 26 topics
\begin{figure}[bp!]
\centering
   \begin{subfigure}[b]{0.7\textwidth}
	   \centering
	   \includegraphics[width=\textwidth]{Figures/alltopics-trump.pdf}
	   %\vspace{-.25cm}
	   \caption{Trump}
	   \label{fig:alltopics-trump} 
	\end{subfigure}
\\
	\begin{subfigure}[b]{0.7\textwidth}
		\centering
	   \includegraphics[width=\textwidth]{Figures/alltopics-bernie.pdf} 
	   \caption{Sanders}
	   \label{fig:alltopics-sanders}
	\end{subfigure}
	\caption{Expected Topic Proportion for All Topics}\label{fig:alltopics}
	%\vspace{-.25cm}
	%{\footnotesize \textit{Note:}}
\end{figure}






% effects of alienation on trump topics
\begin{table}[!ht] \centering 
  \caption{Effects of Electoral Inefficacy and Cynicism on Use of Top 6 Trump Topics}\vspace*{-0.25cm}
  \label{tab:outsider-trump} 
    \renewcommand{\arraystretch}{0.5}
    \begin{adjustbox}{width=\textwidth,center}
\begin{tabular}{@{\extracolsep{5pt}}lcccccc} 
\\[-1.3ex]
\hline\hline \\[-1.8ex] 
\\[-1ex] 
                     & Business   & Political & Head-to-Head & Make America & Bring Back Jobs/ & Change in Gov. \\
                     & Experience & Outsider  & Comparison   & Great Again  & Close Border     &                \\
                     & (1)        & (2)       & (3)          & (4)          & (5)              & (6)            \\ \hline \\
Electoral Inefficacy & 0.022*             & 0.014*   & -0.007    & -0.012            & 0.005    & 0.009*        \\
                     & (-0.008)           & (-0.011) & (-0.012)  & (-0.011)          & (-0.008) & (-0.006)      \\
                     &                    &          &           &                   &          &               \\
Cynicism             & -0.007             & 0.042*   & 0.016     & -0.018*           & 0.011    & -0.007        \\
                     & (-0.01)            & (-0.012) & (-0.014)  & (-0.013)          & (-0.01)  & (-0.008)      \\
                     &                    &          &           &                   &          &               \\
Independent          & 0.018*             & -0.001   & 0.043*    & 0.006             & -0.009   & 0.009         \\
                     & (-0.013)           & (-0.016) & (-0.017)  & (-0.016)          & (-0.012) & (-0.011)      \\
                     &                    &          &           &                   &          &               \\
Republican           & 0.009              & 0.005    & 0.026*    & -0.005            & 0.006    & 0.021*        \\
                     & (-0.009)           & (-0.012) & (-0.013)  & (-0.013)          & (-0.01)  & (-0.008)      \\
                     &                    &          &           &                   &          &               \\
Anti-Immigrant       & -0.008             & -0.006   & 0.000     & 0.003             & 0.006    & -0.004        \\
Attitudes            & (-0.013)           & (-0.018) & (-0.02)   & (-0.018)          & (-0.013) & (-0.012)      \\
                     &                    &          &           &                   &          &               \\
Democratic-Aligned   & -0.013             & 0.009    & 0.006     & -0.003            & -0.019   & 0.009         \\
Group Therm.         & (-0.02)            & (-0.024) & (-0.026)  & (-0.025)          & (-0.02)  & (-0.015)      \\
                     &                    &          &           &                   &          &               \\
Modern Sexism        & -0.013             & -0.029   & 0.011     & 0.012             & 0.013    & -0.005        \\
Index                & (-0.02)            & (-0.03)  & (-0.028)  & (-0.028)          & (-0.021) & (-0.018)      \\
                     &                    &          &           &                   &          &               \\
Child Rearing        & 0.009              & 0.005    & -0.014    & -0.007            & -0.003   & 0.002         \\
Authoritarianism     & (-0.01)            & (-0.013) & (-0.015)  & (-0.013)          & (-0.011) & (-0.008)      \\
                     &                    &          &           &                   &          &               \\
China as             & 0.003              & 0.013    & -0.004    & 0.014             & -0.001   & -0.005        \\
Military Threat      & (-0.01)            & (-0.012) & (-0.013)  & (-0.012)          & (-0.01)  & (-0.009)      \\
                     &                    &          &           &                   &          &               \\
Oppose Trade         & -0.006             & 0.007    & -0.006    & 0.012             & 0.006    & 0.000         \\
                     & (-0.011)           & (-0.013) & (-0.014)  & (-0.013)          & (-0.012) & (-0.008)      \\
                     &                    &          &           &                   &          &               \\
Prospective Econ.    & 0.007              & -0.009   & -0.023    & 0.036*            & 0.01     & 0.000         \\
Assessments          & (-0.012)           & (-0.017) & (-0.019)  & (-0.018)          & (-0.013) & (-0.011)      \\
                     &                    &          &           &                   &          &               \\
Retrospective Econ.  & 0.001              & 0.014    & 0.007     & -0.017            & -0.02*   & 0.002         \\
Assessments          & (-0.012)           & (-0.016) & (-0.017)  & (-0.016)          & (-0.012) & (-0.011)      \\
                     &                    &          &           &                   &          &               \\
Ideology             & -0.017             & -0.013   & 0.038*    & 0.000             & -0.002   & -0.003        \\
                     & (-0.019)           & (-0.023) & (-0.027)  & (-0.023)          & (-0.019) & (-0.015)      \\
                     &                    &          &           &                   &          &               \\
Political Interest   & -0.004             & -0.005   & 0.001     & -0.002            & 0.007    & -0.001        \\
                     & (-0.009)           & (-0.015) & (-0.014)  & (-0.013)          & (-0.01)  & (-0.009)      \\
                     &                    &          &           &                   &          &               \\
Income               & 0.011              & 0.003    & -0.002    & -0.01             & -0.006   & 0.003         \\
                     & (-0.008)           & (-0.013) & (-0.013)  & (-0.011)          & (-0.01)  & (-0.007)      \\
                     &                    &          &           &                   &          &               \\
Education            & -0.009             & 0.012    & 0.003     & 0.004             & -0.001   & -0.002        \\
                     & (-0.011)           & (-0.015) & (-0.017)  & (-0.015)          & (-0.011) & (-0.01)       \\
                     &                    &          &           &                   &          &               \\
Evangelical          & 0.001              & -0.002   & 0.008     & -0.006            & -0.002   & -0.004        \\
                     & (-0.009)           & (-0.012) & (-0.015)  & (-0.012)          & (-0.009) & (-0.008)      \\
                     &                    &          &           &                   &          &               \\
White                & 0.004              & 0.007    & 0.005     & 0.005             & -0.004   & 0.005         \\
                     & (-0.008)           & (-0.011) & (-0.012)  & (-0.011)          & (-0.009) & (-0.007)      \\
                     &                    &          &           &                   &          &               \\
Female               & 0.001              & -0.008   & 0.002     & 0.004             & 0.001    & -0.002        \\
                     & (-0.006)           & (-0.008) & (-0.009)  & (-0.007)          & (-0.006) & (-0.005)      \\
                     &                    &          &           &                   &          &               \\
Age                  & 0.000              & 0.020    & -0.011    & 0.003             & -0.001   & 0.019*        \\
                     & (-0.012)           & (-0.017) & (-0.018)  & (-0.017)          & (-0.013) & (-0.012)      \\
                     &                    &          &           &                   &          &               \\
Constant             & 0.057*             & -0.005   & 0.006     & 0.044             & 0.036    & 0.016         \\
                     & (-0.027)           & (-0.036) & (-0.041)  & (-0.037)          & (-0.028) & (-0.024)      \\ \hline
Observations         & 1,099      		  & 1,099    & 1,099     & 1,099          	 &	1,099   &  1,099  		\\
\hline 
\hline \\[-1.5ex] 
\multicolumn{7}{l}{\footnotesize $^{*}$p$<$0.1; $^{**}$p$<$0.05; $^{***}$p$<$0.01; one-tailed tests} \\ 
\multicolumn{7}{l}{\footnotesize Standard errors in parentheses} \\
\multicolumn{7}{l}{\footnotesize All variables scaled to range between 0 and 1} \\
%\multicolumn{2}{l}{\footnotesize Reference category for `Republican' and} \\
%\multicolumn{2}{l}{\footnotesize \hspace{2pt} `Independent' is `Democrat'.} \\
\multicolumn{7}{l}{\footnotesize Leaners are included as partisans.} \\
%\multicolumn{2}{l}{\footnotesize }
\end{tabular} 
\end{adjustbox}
\end{table}



\clearpage



% effects of alienation on sanders topics
\begin{table}[!ht] \centering 
  \caption{Effects of Electoral Inefficacy and Cynicism on Use of Top 6 Sanders Topics}\vspace*{-0.25cm}
  \label{tab:outsider-bernie} 
  \renewcommand{\arraystretch}{0.5}
      \begin{adjustbox}{width=\textwidth,center}
\begin{tabular}{@{\extracolsep{5pt}}lcccccc} 
\\[-1.3ex]
\hline\hline \\[-1.8ex] 
\\[-1ex] 
                     & Care About People/   & Ideas/  & Political & He's Not Clinton/ & Economy/ & Takes A Stand \\
                     & Representation       & Agenda  & Outsider  & Want Change       & Progress &               \\
                     & (1)        & (2)       & (3)          & (4)          & (5)              & (6)            \\ \hline \\
Electoral Inefficacy & 0.104*             & -0.115*  & -0.055    & -0.005            & -0.054*  & 0.076*        \\
                     & (-0.048)           & (-0.039) & (-0.044)  & (-0.024)          & (-0.04)  & (-0.03)       \\
                     &                    &          &           &                   &          &               \\
Political System     & 0.145*             & -0.089   & 0.134*    & -0.065            & 0.089    & 0.139*        \\
Illegitimacy         & (-0.099)           & (-0.075) & (-0.091)  & (-0.056)          & (-0.08)  & (-0.061)      \\
                     &                    &          &           &                   &          &               \\
Partisan Strength    & 0.008              & 0.015    & -0.013    & 0.003             & -0.009   & 0.002         \\
                     & (-0.037)           & (-0.03)  & (-0.038)  & (-0.021)          & (-0.031) & (-0.025)      \\
                     &                    &          &           &                   &          &               \\
Ideology             & -0.082             & 0.035    & -0.068    & 0.051             & -0.01    & 0.055         \\
                     & (-0.088)           & (-0.065) & (-0.078)  & (-0.046)          & (-0.073) & (-0.058)      \\
                     &                    &          &           &                   &          &               \\
Sociotropic Econ.    & 0.031              & -0.018   & 0.006     & -0.006            & 0.001    & -0.004        \\
Assessments          & (-0.063)           & (-0.048) & (-0.056)  & (-0.034)          & (-0.051) & (-0.043)      \\
                     &                    &          &           &                   &          &               \\
Personal Econ.       & -0.077             & 0.003    & 0.061     & -0.029            & 0.015    & -0.035        \\
Assessments          & (-0.068)           & (-0.055) & (-0.059)  & (-0.039)          & (-0.06)  & (-0.05)       \\
                     &                    &          &           &                   &          &               \\
Social Dominance     & -0.01              & 0.023    & 0.068     & -0.008            & -0.051   & -0.007        \\
Orientation          & (-0.104)           & (-0.08)  & (-0.087)  & (-0.051)          & (-0.079) & (-0.07)       \\
                     &                    &          &           &                   &          &               \\
Modern Sexism        & 0.052              & 0.01     & -0.005    & 0.025             & -0.015   & -0.018        \\
Index                & (-0.087)           & (-0.074) & (-0.08)   & (-0.05)           & (-0.071) & (-0.063)      \\
                     &                    &          &           &                   &          &               \\
Age                  & -0.092             & -0.02    & -0.006    & 0.01              & -0.007   & -0.015        \\
                     & (-0.106)           & (-0.078) & (-0.087)  & (-0.054)          & (-0.077) & (-0.065)      \\
                     &                    &          &           &                   &          &               \\
Education            & -0.044             & 0.005    & 0.026     & 0.007             & 0.007    & -0.006        \\
                     & (-0.065)           & (-0.049) & (-0.056)  & (-0.033)          & (-0.054) & (-0.04)       \\
                     &                    &          &           &                   &          &               \\
Income               & 0.002              & -0.025   & -0.031    & 0.01              & 0.009    & -0.01         \\
                     & (-0.051)           & (-0.038) & (-0.043)  & (-0.026)          & (-0.042) & (-0.035)      \\
                     &                    &          &           &                   &          &               \\
Black                & 0.02               & -0.028   & 0.035     & -0.013            & -0.021   & -0.002        \\
                     & (-0.068)           & (-0.043) & (-0.053)  & (-0.032)          & (-0.045) & (-0.045)      \\
                     &                    &          &           &                   &          &               \\
White                & -0.03              & -0.003   & 0.02      & -0.002            & 0.01     & -0.029        \\
                     & (-0.045)           & (-0.032) & (-0.04)   & (-0.021)          & (-0.031) & (-0.031)      \\
                     &                    &          &           &                   &          &               \\
Female               & 0.001              & 0.009    & -0.037*   & 0.009             & 0.006    & 0.015         \\
                     & (-0.035)           & (-0.024) & (-0.028)  & (-0.017)          & (-0.024) & (-0.022)      \\
                     &                    &          &           &                   &          &               \\
Constant             & 0.113              & 0.2*     & 0.036     & 0.095             & 0.063    & -0.001        \\
                     & (-0.15)            & (-0.109) & (-0.135)  & (-0.085)          & (-0.124) & (-0.092)      \\ \hline
Observations         & 174        		  & 174      & 174       & 174          	 &	174     &  174  		\\
\hline 
\hline \\[-1.5ex] 
\multicolumn{7}{l}{\footnotesize $^{*}$p$<$0.1; $^{**}$p$<$0.05; $^{***}$p$<$0.01; one-tailed tests} \\ 
\multicolumn{7}{l}{\footnotesize Standard errors in parentheses} \\
\multicolumn{7}{l}{\footnotesize All variables scaled to range between 0 and 1} \\
%\multicolumn{2}{l}{\footnotesize Reference category for `Republican' and} \\
%\multicolumn{2}{l}{\footnotesize \hspace{2pt} `Independent' is `Democrat'.} \\
\multicolumn{7}{l}{\footnotesize Leaners are included as partisans.} \\
%\multicolumn{2}{l}{\footnotesize }
\end{tabular} 
\end{adjustbox}
\end{table}




\section{Modeling responses on other candidates}\label{app:model-responses-other-candidates}
A potential concern regarding my analysis of open-ended responses about Trump and Sanders is that it does not reveal whether being perceived as an ``outsider" is a phenomenon unique to these two candidates. Surely Trump and Sanders are not the only outsiders in the history of U.S. presidential elections (e.g., Ross Perot in 1992), but if their outsider personas were truly responsible for capturing the support of the politically alienated in 2016, then there should be no mentions of ``outsider" qualities when the public is asked to evaluate candidates other than Trump and Sanders in that election.

I perform two exercises to show that perceptions as an outsider were less relevant for candidates other than Trump and Sanders. First, on the Republican side, I rely on the open-ended responses on the things that Republicans respondents liked about their most preferred candidate in their party's primary from the ISCAP panel. This means that I combined all responses about candidates in the Republican primary into a single corpus, including Ben Carson, Carly Fiorina, Chris Christie, Jeb Bush, John Kasich, Marco Rubio, Mike Huckabee, Rand Paul and Ted Cruz ($N$=258). In essence this forms a corpus of texts representing the things that Republicans like about their preferred candidate in the primary (not including Trump), and the expectation is that there should be no topic dedicated to candidates' outsider qualities (or if there is, its expected frequency should be small).

Figure~\ref{fig:alltopics-reps} shows the expected topic proportions for all topics from a 15-topic model of the texts about Republican primary candidates other than Trump. Noticeably, there is no topic dedicated to candidates outsider qualities like we saw with Trump (Figure~\ref{fig:top6topics-trump}). In fact, there is actually a topic dedicated to candidates experience in politics (Topic 10), which is the conceptual opposite of being an outsider. This descriptive analysis clearly shows that, at least on the Republican side, it was Trump that was uniquely perceived as an outsider. 

\begin{figure}[h!]
	\centering
	\includegraphics[width=\textwidth]{Figures/alltopics-reps}
	\caption{Expected Topic Proportion for All Topics, Candidates in 2016 Republican Primary Except Trump, ISCAP}\label{fig:alltopics-reps}
\end{figure}

\begin{figure}[h!]
	\centering
	\includegraphics[width=\textwidth]{Figures/alltopics-clinton}
	\caption{Expected Topic Proportion for All Topics, Hillary Clinton, ANES}
		\label{fig:alltopics-clinton}
\end{figure}

Next, on the Democratic side, I use the open-ended responses on the things that all ANES respondents liked about Hillary Clinton ($N = 1,148$). Here again the expectation is that Clinton should not be described as an outsider---a reasonable expectation given that she was one of the most qualified candidates to ever seek the office. Figure~\ref{fig:alltopics-clinton} shows the expected topic proportion for all topics from a 28-topic model of the texts about Clinton, and as expected, respondents largely viewed the former Senator as connected to and experienced in politics. Topics 9 and 14, for instance, use words such as `experi(ence),' `servic(e),' `record,' `profess(ional),' and `compet(ant/ance)' to describe Clinton. Importantly, there is no topic that depicts Clinton as an outsider opposed to the current political order as there was in the Sanders models. This is again consistent with the idea that candidates like Trump and Sanders were uniquely able to capture the support of the political alienated specifically because these candidates were seen as outsiders. 







%%%%%%%%%%%%%%%%%%%%%%%%%%%%%%%%%%%%%%%%%%%%%%%%%%%%
%%  %%%%%  %%%%%  %%%%%  %%%%%  %  %%%%%  %    %  %%
%%  %      %      %        %    %  %   %  % %  %  %%
%%  %%%%%  %%%%%  %        %    %  %   %  %  % %  %%
%%      %  %      %        %    %  %   %  %   %%  %%
%%  %%%%%  %%%%%  %%%%%    %    %  %%%%%  %    %  %%
%%%%%%%%%%%%%%%%%%%%%%%%%%%%%%%%%%%%%%%%%%%%%%%%%%%%





%%%%%%%%%%%%%%%%%%%%%%%%%%%%%%%%%%%%%%%%%%%%%%%%%%%%
%%  %%%%%  %%%%%  %%%%%  %%%%%  %  %%%%%  %    %  %%
%%  %      %      %        %    %  %   %  % %  %  %%
%%  %%%%%  %%%%%  %        %    %  %   %  %  % %  %%
%%      %  %      %        %    %  %   %  %   %%  %%
%%  %%%%%  %%%%%  %%%%%    %    %  %%%%%  %    %  %%
%%%%%%%%%%%%%%%%%%%%%%%%%%%%%%%%%%%%%%%%%%%%%%%%%%%%




% VOTING BEHAVIOR MODELS
\clearpage
\section{Models of Voting Behavior}\label{app:votemodels}

% Turnout (1988-2016)
\begin{table}[!htbp] \centering 
  \caption{Turnout in the 1988-2016 U.S. Presidential Elections}\label{tab:turnout} 
	\begin{adjustbox}{width=\textwidth,center}
	\begin{tabular}{@{\extracolsep{5pt}}lcccccccc} 
	\\[-1.8ex]\hline 
	\hline \\[-1.8ex] 
	 & \multicolumn{8}{c}{\textit{Dependent variable:}} \\ 
	\cline{2-9} 
	 & 1988 & 1992 & 1996 & 2000 & 2004 & 2008 & 2012 & 2016 \\ 
	\\[-1.8ex] & (1) & (2) & (3) & (4) & (5) & (6) & (7) & (8)\\ 
	\hline \\[-1.8ex] 
 Electoral Ineff. & $-$0.222 & $-$0.182 & $-$0.176 & $-$0.635$^{***}$ & $-$0.334 & $-$1.102$^{***}$ & $-$0.223 & $-$0.088 \\ 
  & (0.198) & (0.187) & (0.225) & (0.228) & (0.310) & (0.386) & (0.181) & (0.105) \\ 
 Cynicism & $-$0.157 & $-$0.262$^{*}$ & $-$0.133 & $-$0.089 & $-$0.155 & 0.212 & 0.053 & 0.396$^{***}$ \\ 
  & (0.186) & (0.185) & (0.213) & (0.205) & (0.247) & (0.334) & (0.175) & (0.109) \\ 
 Independent & $-$0.479$^{**}$ & $-$0.843$^{***}$ & $-$0.781$^{***}$ & $-$0.857$^{***}$ & $-$0.969$^{***}$ & $-$1.053$^{***}$ & $-$0.948$^{***}$ & $-$0.844$^{***}$ \\ 
  & (0.226) & (0.189) & (0.250) & (0.232) & (0.293) & (0.381) & (0.167) & (0.111) \\ 
 Republican & 0.085 & $-$0.334$^{**}$ & 0.154 & 0.097 & 0.110 & $-$0.065 & 0.070 & $-$0.103 \\ 
  & (0.153) & (0.144) & (0.171) & (0.176) & (0.225) & (0.315) & (0.143) & (0.083) \\ 
 Education & 2.165$^{***}$ & 2.281$^{***}$ & 1.973$^{***}$ & 1.746$^{***}$ & 1.636$^{***}$ & 1.720$^{***}$ & 1.511$^{***}$ & 1.115$^{***}$ \\ 
  & (0.277) & (0.269) & (0.302) & (0.305) & (0.386) & (0.510) & (0.199) & (0.127) \\ 
 Political Interest & 2.116$^{***}$ & 1.801$^{***}$ & 2.139$^{***}$ & 1.819$^{***}$ & 1.793$^{***}$ & 1.314$^{***}$ & 1.471$^{***}$ & 0.890$^{***}$ \\ 
  & (0.210) & (0.189) & (0.236) & (0.244) & (0.279) & (0.380) & (0.174) & (0.108) \\ 
 Income & 1.767$^{***}$ & 1.675$^{***}$ & 1.686$^{***}$ & 1.351$^{***}$ & 1.050$^{***}$ & 0.515 & 0.747$^{***}$ & 0.352$^{***}$ \\ 
  & (0.274) & (0.244) & (0.301) & (0.309) & (0.351) & (0.495) & (0.190) & (0.112) \\ 
 White & $-$0.327$^{*}$ & 0.391$^{**}$ & 0.060 & 0.528$^{***}$ & 0.646$^{***}$ & 0.156 & 0.346$^{**}$ & 0.320$^{***}$ \\ 
  & (0.221) & (0.194) & (0.233) & (0.225) & (0.265) & (0.307) & (0.191) & (0.094) \\ 
 Black & $-$0.299 & 0.415$^{**}$ & 0.126 & 0.861$^{***}$ & 0.641$^{**}$ & 0.542$^{*}$ & 1.022$^{***}$ & 0.530$^{***}$ \\ 
  & (0.272) & (0.247) & (0.307) & (0.321) & (0.336) & (0.386) & (0.240) & (0.144) \\ 
 Age & 3.028$^{***}$ & 2.496$^{***}$ & 2.795$^{***}$ & 1.877$^{***}$ & 0.906$^{**}$ & 1.915$^{***}$ & 2.397$^{***}$ & 0.883$^{***}$ \\ 
  & (0.358) & (0.321) & (0.396) & (0.408) & (0.460) & (0.647) & (0.273) & (0.155) \\ 
 Constant & $-$2.419$^{***}$ & $-$2.313$^{***}$ & $-$2.436$^{***}$ & $-$2.058$^{***}$ & $-$1.652$^{***}$ & $-$1.052$^{**}$ & $-$1.742$^{***}$ & $-$1.618$^{***}$ \\ 
  & (0.310) & (0.289) & (0.361) & (0.359) & (0.405) & (0.562) & (0.268) & (0.154) \\ 
\hline \\[-1.8ex] 
Observations & 1,483 & 1,907 & 1,329 & 1,231 & 890 & 453 & 2,386 & 3,909 \\ 
Akaike Inf. Crit. & 1,373.297 & 1,617.800 & 1,151.529 & 1,074.497 & 740.082 & 415.487 & 1,848.943 & 4,601.912 \\ 
	\hline 
	\hline \\[-1.8ex] 
	\multicolumn{9}{l}{\footnotesize $^{*}$p$<$0.1; $^{**}$p$<$0.05; $^{***}$p$<$0.01; one-tailed tests} \\ 
	\multicolumn{9}{l}{\footnotesize These regression estimates used to produce Figure~\ref{fig:turnoutcoef-cynicism}.}
	\end{tabular} 
	\end{adjustbox}
\end{table} 
\clearpage


% Vote Choice in Primary Election
\begin{table}[!t] \centering 
  \caption{Effect of Cynicism and Election Unresponsiveness on Vote Choice in 2016 Primary Election, Multinomial Logit} 
  \label{tab:primary} 
   \renewcommand{\arraystretch}{1}
\begin{adjustbox}{width=0.6\textwidth,center}
\begin{tabular}{@{\extracolsep{5pt}}lccc} 
\\[-1.8ex]\hline 
 \\[-2ex] 
 & Sanders & Trump & Did Not Vote \\ 
\hline \\[-1.8ex] 
\hline \\[-1.8ex] 
 Electoral Inefficacy & $-$0.971 & 0.210 & $-$0.102 \\ 
  & (0.761) & (0.544) & (0.383) \\ 
 Cynicism & 1.665$^{**}$ & 0.157 & $-$0.152 \\ 
  & (0.800) & (0.688) & (0.412) \\ 
 Republican & $-$2.132$^{**}$ & 16.676$^{***}$ & 0.445 \\ 
  & (1.017) & (0.683) & (0.410) \\ 
 Independent & $-$0.476 & 15.965$^{***}$ & $-$0.392 \\ 
  & (0.774) & (0.711) & (0.548) \\ 
 Political Interest & 1.450$^{**}$ & 1.647$^{***}$ & 1.083$^{***}$ \\ 
  & (0.805) & (0.658) & (0.443) \\ 
 Ideology & $-$1.418 & 0.536 & 0.446 \\ 
  & (1.282) & (1.119) & (0.707) \\ 
 Anti-Immigrant Attitudes & 0.583 & 1.186$^{*}$ & 0.506 \\ 
  & (1.113) & (0.836) & (0.597) \\ 
 Democratic-Aligned Group Therm. & 0.853 & $-$3.011$^{**}$ & 1.043 \\ 
  & (1.611) & (1.291) & (0.871) \\ 
 Child-Rearing Authoritarianism & $-$2.081$^{**}$ & $-$0.887 & $-$1.138$^{**}$ \\ 
  & (0.952) & (0.699) & (0.486) \\ 
 Modern Sexism Index & 0.247 & 0.314 & 0.526 \\ 
  & (2.089) & (1.302) & (0.919) \\ 
 China as Military Threat & $-$0.518 & $-$0.462 & 0.394 \\ 
  & (0.684) & (0.602) & (0.405) \\ 
 Oppose Trade & 0.328 & $-$0.647 & $-$1.563$^{***}$ \\ 
  & (0.963) & (0.623) & (0.494) \\ 
 Prospective Econ. Assessments & $-$0.647 & 1.870$^{**}$ & $-$0.006 \\ 
  & (1.295) & (0.828) & (0.607) \\ 
 Retrospective Econ. Assessments & $-$0.864 & $-$1.773$^{**}$ & $-$0.786$^{*}$ \\ 
  & (1.124) & (0.860) & (0.604) \\ 
 Income & 0.318 & 0.238 & 0.760$^{**}$ \\ 
  & (0.705) & (0.563) & (0.403) \\ 
 Education & 1.321$^{*}$ & 2.871$^{***}$ & 1.199$^{**}$ \\ 
  & (1.011) & (0.837) & (0.510) \\ 
 Evangelical & $-$25.520$^{***}$ & 0.181 & 0.720$^{*}$ \\ 
  & (0.000) & (0.543) & (0.446) \\ 
 White & 0.128 & 0.478 & 0.135 \\ 
  & (0.668) & (0.538) & (0.356) \\ 
 Black & 1.563$^{**}$ & $-$10.514$^{***}$ & 1.651$^{***}$ \\ 
  & (0.833) & (0.00000) & (0.537) \\ 
 Female & $-$0.381 & 0.353 & 0.383$^{*}$ \\ 
  & (0.471) & (0.382) & (0.261) \\ 
 Age & $-$1.107 & 3.467$^{***}$ & 2.792$^{***}$ \\ 
  & (1.196) & (0.868) & (0.599) \\ 
 Constant & $-$2.858 & $-$22.267$^{***}$ & $-$4.591$^{***}$ \\ 
  & (2.299) & (1.239) & (1.203) \\ 
\hline \\[-1.8ex] 
Akaike Inf. Crit. & 906.966 & 906.966 & 906.966 \\ 
Observations =  & 482 & 482 & 482 \\
\hline 
\hline \\[-1.8ex] 
\multicolumn{4}{l}{\footnotesize $^{*}$p$<$0.1; $^{**}$p$<$0.05; $^{***}$p$<$0.01; one-tailed tests} \\ 
\multicolumn{4}{l}{\footnotesize All predictors range between 0 and 1} \\
\multicolumn{4}{l}{\footnotesize Reference category for dependent variables is `Any candidate other} \\
\multicolumn{4}{l}{\footnotesize \hspace{2pt} than Trump or Clinton'} \\
\multicolumn{4}{l}{\footnotesize Analysis limited to Super Tuesday states with open primaries}
\end{tabular} 
\end{adjustbox}
\end{table}




% Vote Choice in General Election
\begin{table}[!t] \centering 
  \caption{Effect of Cynicism and Election Unresponsiveness on Vote Choice in 2016 General Election, Multinomial Logit} 
  \label{tab:general} 
   \renewcommand{\arraystretch}{1}
\begin{adjustbox}{width=0.6\textwidth,center}
\begin{tabular}{@{\extracolsep{5pt}}lccc} 
\\[-1.8ex]\hline 
 \\[-2ex] 
 & Clinton & Trump & Other \\ 
\hline \\[-1.8ex] 
\hline \\[-1.8ex] 
 Electoral Inefficacy & 0.105 & $-$0.117 & 0.225 \\ 
  & (0.218) & (0.224) & (0.293) \\ 
 Cynicism & $-$0.006 & 0.577$^{**}$ & 0.714$^{**}$ \\ 
  & (0.226) & (0.246) & (0.322) \\ 
 Republican & $-$1.665$^{***}$ & 1.745$^{***}$ & 0.429$^{*}$ \\ 
  & (0.223) & (0.230) & (0.283) \\ 
 Independent & $-$1.574$^{***}$ & 0.301 & $-$0.033 \\ 
  & (0.228) & (0.270) & (0.300) \\ 
 Political Interest & 1.256$^{***}$ & 1.219$^{***}$ & 0.851$^{***}$ \\ 
  & (0.224) & (0.233) & (0.299) \\ 
 Ideology & $-$0.335 & 2.686$^{***}$ & 1.027$^{**}$ \\ 
  & (0.410) & (0.450) & (0.574) \\ 
 Anti-Immigrant Attitudes & $-$1.537$^{***}$ & 1.521$^{***}$ & $-$0.929$^{**}$ \\ 
  & (0.343) & (0.343) & (0.464) \\ 
 Democratic-Aligned Group Therm. & 1.110$^{**}$ & 0.175 & 0.991$^{*}$ \\ 
  & (0.494) & (0.508) & (0.660) \\ 
 Child-Rearing Authoritarianism & $-$0.957$^{***}$ & $-$0.108 & $-$0.745$^{**}$ \\ 
  & (0.266) & (0.280) & (0.363) \\ 
 Modern Sexism Index & 1.191$^{**}$ & $-$0.241 & $-$0.235 \\ 
  & (0.545) & (0.543) & (0.748) \\ 
 China as Military Threat & 0.154 & 0.099 & $-$0.376 \\ 
  & (0.225) & (0.242) & (0.305) \\ 
 Oppose Trade & $-$0.304 & 0.983$^{***}$ & 0.182 \\ 
  & (0.286) & (0.292) & (0.388) \\ 
 Prospective Econ. Assessments & 0.013 & 0.471$^{*}$ & $-$0.481 \\ 
  & (0.362) & (0.358) & (0.483) \\ 
 Retrospective Econ. Assessments & 1.826$^{***}$ & $-$0.334 & 1.150$^{***}$ \\ 
  & (0.343) & (0.346) & (0.466) \\ 
 Income & 0.894$^{***}$ & 0.526$^{**}$ & 0.437$^{*}$ \\ 
  & (0.235) & (0.243) & (0.316) \\ 
 Education & 1.087$^{***}$ & 0.984$^{***}$ & 0.971$^{***}$ \\ 
  & (0.279) & (0.288) & (0.393) \\ 
 Evangelical & 0.518$^{*}$ & 0.459$^{*}$ & 0.599$^{*}$ \\ 
  & (0.354) & (0.306) & (0.397) \\ 
 White & 0.265$^{*}$ & 0.920$^{***}$ & 0.508$^{**}$ \\ 
  & (0.185) & (0.216) & (0.269) \\ 
 Black & 1.329$^{***}$ & 0.356 & 0.962$^{**}$ \\ 
  & (0.304) & (0.514) & (0.471) \\ 
 Female & 0.224$^{*}$ & 0.363$^{**}$ & 0.270$^{*}$ \\ 
  & (0.152) & (0.157) & (0.204) \\ 
 Age & 2.144$^{***}$ & 2.116$^{***}$ & 0.226 \\ 
  & (0.339) & (0.346) & (0.467) \\ 
 Constant & $-$3.239$^{***}$ & $-$7.411$^{***}$ & $-$4.444$^{***}$ \\ 
  & (0.697) & (0.758) & (0.958) \\ 
\hline \\[-1.8ex] 
Akaike Inf. Crit. & 3,523.970 & 3,523.970 & 3,523.970 \\ 
Observations =  & 2,417 & 2,417 & 2,417 \\
\hline 
\hline \\[-1.8ex] 
\multicolumn{4}{l}{\footnotesize $^{*}$p$<$0.1; $^{**}$p$<$0.05; $^{***}$p$<$0.01; one-tailed tests} \\ 
\multicolumn{4}{l}{\footnotesize All predictors range between 0 and 1} \\
\multicolumn{4}{l}{\footnotesize Reference category for the dependent variables is `Any candidate other} \\
\multicolumn{4}{l}{\footnotesize \hspace{2pt} than Trump or Clinton'} \\
%\multicolumn{4}{l}{\footnotesize Reference category for `Independent' and `Republican' is `Democrat'} \\
\end{tabular} 
\end{adjustbox}
\end{table}



% Vote Choice in Primary Election - Pre-election Variables Only
\begin{table}[!t] \centering 
  \caption{Effect of Cynicism and Election Unresponsiveness on Vote Choice in 2016 Primary Election, Pre-election Variables Only, Multinomial Logit} 
  \label{tab:primary-pre} 
   \renewcommand{\arraystretch}{1}
\begin{adjustbox}{width=0.6\textwidth,center}
\begin{tabular}{@{\extracolsep{5pt}}lccc} 
\\[-1.8ex]\hline 
 \\[-2ex] 
 & Sanders & Trump & Did Not Vote \\ 
\hline \\[-1.8ex] 
\hline \\[-1.8ex] 
 Electoral Inefficacy & $-$0.575 & 0.096 & $-$0.057 \\ 
  & (0.693) & (0.524) & (0.371) \\ 
 Cynicism & 1.552$^{**}$ & 0.119 & $-$0.107 \\ 
  & (0.755) & (0.670) & (0.403) \\ 
 Republican & $-$1.875$^{**}$ & 15.745$^{***}$ & 0.280 \\ 
  & (0.940) & (0.577) & (0.394) \\ 
 Independent & $-$0.260 & 14.983$^{***}$ & $-$0.495 \\ 
  & (0.739) & (0.613) & (0.533) \\ 
 Political Interest & 1.416$^{**}$ & 1.332$^{**}$ & 1.198$^{***}$ \\ 
  & (0.758) & (0.627) & (0.419) \\ 
 Ideology & $-$2.184$^{**}$ & 0.379 & 0.098 \\ 
  & (1.200) & (1.083) & (0.666) \\ 
 Anti-Immigrant Attitudes & $-$0.086 & 1.279$^{*}$ & 0.014 \\ 
  & (1.083) & (0.771) & (0.564) \\ 
 Prospective Econ. Assessments & $-$0.375 & 1.397$^{**}$ & 0.189 \\ 
  & (1.134) & (0.774) & (0.578) \\ 
 Retrospective Econ. Assessments & $-$0.583 & $-$1.597$^{**}$ & $-$0.491 \\ 
  & (1.057) & (0.820) & (0.584) \\ 
 Income & 0.341 & 0.147 & 0.989$^{***}$ \\ 
  & (0.654) & (0.541) & (0.392) \\ 
 Education & 1.696$^{**}$ & 2.872$^{***}$ & 1.515$^{***}$ \\ 
  & (0.961) & (0.770) & (0.498) \\ 
 Evangelical & $-$27.583$^{***}$ & 0.050 & 0.652$^{*}$ \\ 
  & (0.000) & (0.522) & (0.430) \\ 
 White & 0.625 & 0.428 & 0.249 \\ 
  & (0.631) & (0.516) & (0.341) \\ 
 Black & 1.437$^{**}$ & $-$14.929$^{***}$ & 1.634$^{***}$ \\ 
  & (0.803) & (0.00000) & (0.527) \\ 
 Female & $-$0.435 & 0.063 & 0.452$^{**}$ \\ 
  & (0.449) & (0.350) & (0.248) \\ 
 Age & $-$1.620$^{*}$ & 3.354$^{***}$ & 2.613$^{***}$ \\ 
  & (1.136) & (0.824) & (0.579) \\ 
 Constant & $-$3.601$^{**}$ & $-$23.132$^{***}$ & $-$5.007$^{***}$ \\ 
  & (1.546) & (1.024) & (0.901) \\ 
\hline \\[-1.8ex] 
Akaike Inf. Crit. & 910.968 & 910.968 & 910.968 \\ 
Observations =  & 482 & 482 & 482 \\
\hline 
\hline \\[-1.8ex] 
\multicolumn{4}{l}{\footnotesize $^{*}$p$<$0.1; $^{**}$p$<$0.05; $^{***}$p$<$0.01; one-tailed tests} \\ 
\multicolumn{4}{l}{\footnotesize All predictors range between 0 and 1} \\
\multicolumn{4}{l}{\footnotesize Reference category for dependent variables is `Any candidate other} \\
\multicolumn{4}{l}{\footnotesize \hspace{2pt} than Trump or Clinton'} \\
\multicolumn{4}{l}{\footnotesize Analysis limited to Super Tuesday states with open primaries}
\end{tabular} 
\end{adjustbox}
\end{table}




% Vote Choice in General Election - Pre-election Variables Only
\begin{table}[!t] \centering 
  \caption{Effect of Cynicism and Election Unresponsiveness on Vote Choice in 2016 General Election, Pre-election Variables Only, Multinomial Logit} 
  \label{tab:general-pre} 
   \renewcommand{\arraystretch}{1}
\begin{adjustbox}{width=0.6\textwidth,center}
\begin{tabular}{@{\extracolsep{5pt}}lccc} 
\\[-1.8ex]\hline 
 \\[-2ex] 
 & Clinton & Trump & Other \\ 
\hline \\[-1.8ex] 
\hline \\[-1.8ex] 
 Electoral Inefficacy & 0.067 & $-$0.108 & 0.218 \\ 
  & (0.217) & (0.222) & (0.292) \\ 
 Cynicism & 0.056 & 0.619$^{***}$ & 0.764$^{**}$ \\ 
  & (0.222) & (0.243) & (0.320) \\ 
 Republican & $-$1.646$^{***}$ & 1.701$^{***}$ & 0.419$^{*}$ \\ 
  & (0.220) & (0.227) & (0.279) \\ 
 Independent & $-$1.508$^{***}$ & 0.281 & 0.005 \\ 
  & (0.224) & (0.267) & (0.298) \\ 
 Political Interest & 1.340$^{***}$ & 1.256$^{***}$ & 0.943$^{***}$ \\ 
  & (0.221) & (0.228) & (0.295) \\ 
 Ideology & $-$0.756$^{**}$ & 2.625$^{***}$ & 0.627 \\ 
  & (0.394) & (0.440) & (0.551) \\ 
 Anti-Immigrant Attitudes & $-$1.937$^{***}$ & 1.601$^{***}$ & $-$1.261$^{***}$ \\ 
  & (0.326) & (0.323) & (0.441) \\ 
 Prospective Econ. Assessments & 0.041 & 0.423 & $-$0.466 \\ 
  & (0.358) & (0.353) & (0.476) \\ 
 Retrospective Econ. Assessments & 1.945$^{***}$ & $-$0.394 & 1.236$^{***}$ \\ 
  & (0.340) & (0.341) & (0.459) \\ 
 Income & 0.982$^{***}$ & 0.466$^{**}$ & 0.537$^{**}$ \\ 
  & (0.231) & (0.238) & (0.312) \\ 
 Education & 1.309$^{***}$ & 0.889$^{***}$ & 1.190$^{***}$ \\ 
  & (0.269) & (0.276) & (0.381) \\ 
 Evangelical & 0.435 & 0.399$^{*}$ & 0.530$^{*}$ \\ 
  & (0.351) & (0.303) & (0.393) \\ 
 White & 0.321$^{**}$ & 0.944$^{***}$ & 0.569$^{**}$ \\ 
  & (0.179) & (0.213) & (0.265) \\ 
 Black & 1.350$^{***}$ & 0.310 & 0.919$^{**}$ \\ 
  & (0.303) & (0.507) & (0.468) \\ 
 Female & 0.223$^{*}$ & 0.346$^{**}$ & 0.261$^{*}$ \\ 
  & (0.148) & (0.154) & (0.201) \\ 
 Age & 1.921$^{***}$ & 2.006$^{***}$ & $-$0.036 \\ 
  & (0.327) & (0.337) & (0.457) \\ 
 Constant & $-$2.490$^{***}$ & $-$6.791$^{***}$ & $-$4.410$^{***}$ \\ 
  & (0.463) & (0.530) & (0.654) \\ 
\hline \\[-1.8ex] 
Akaike Inf. Crit. & 3,548.554 & 3,548.554 & 3,548.554 \\ 
Observations =  & 2,417 & 2,417 & 2,417 \\
\hline 
\hline \\[-1.8ex] 
\multicolumn{4}{l}{\footnotesize $^{*}$p$<$0.1; $^{**}$p$<$0.05; $^{***}$p$<$0.01; one-tailed tests} \\ 
\multicolumn{4}{l}{\footnotesize All predictors range between 0 and 1} \\
\multicolumn{4}{l}{\footnotesize Reference category for the dependent variables is `Any candidate other} \\
\multicolumn{4}{l}{\footnotesize \hspace{2pt} than Trump or Clinton'} \\
%\multicolumn{4}{l}{\footnotesize Reference category for `Independent' and `Republican' is `Democrat'} \\
\end{tabular} 
\end{adjustbox}
\end{table}


%% Alienation on Two-Party Vote
%\begin{table}[!ht] \centering 
%  \caption{Effects of Electoral Inefficacy and Cynicism on 2008 Two-Party Vote, Logit}\vspace*{-0.25cm}
%  \label{tab:obamavote} 
%    \renewcommand{\arraystretch}{0.7}
%\begin{tabular}{@{\extracolsep{5pt}}lc} 
%\\[-1.3ex]
%\hline\hline \\[-1.8ex] 
%\\[-1ex] 
%& Obama Vote \\
%&   \\[0.5ex]
%& (1) \\[0.5ex]
%\hline \\[-0.5ex] 
% Electoral Ineff. & 0.608$^{*}$ \\ 
%  & (0.465) \\ 
%  & \\ 
% Cynicism & $-$0.095 \\ 
%  & (0.385) \\ 
%  & \\ 
% Independent & $-$2.979$^{***}$ \\ 
%  & (0.433) \\ 
%  & \\ 
% Republican & $-$4.240$^{***}$ \\ 
%  & (0.330) \\ 
%  & \\ 
% Education & 0.150 \\ 
%  & (0.548) \\ 
%  & \\ 
% Income & $-$0.688 \\ 
%  & (0.570) \\ 
%  & \\ 
% White & $-$1.229$^{***}$ \\ 
%  & (0.334) \\ 
%  & \\ 
% Black & 3.302$^{***}$ \\ 
%  & (1.067) \\ 
%  & \\ 
% Age & $-$1.791$^{***}$ \\ 
%  & (0.725) \\ 
%  & \\ 
% Constant & 3.929$^{***}$ \\ 
%  & (0.653) \\ 
%  & \\ 
%\hline \\[-1.8ex] 
%Observations & 227 \\ 
%Akaike Inf. Crit. & 365.961 \\ 
%\hline 
%\hline \\[-1.8ex] 
%\multicolumn{2}{l}{\footnotesize $^{*}$p$<$0.1; $^{**}$p$<$0.05; $^{***}$p$<$0.01; one-tailed tests} \\ 
%\multicolumn{2}{l}{\footnotesize Standard errors in parentheses} \\
%\multicolumn{2}{l}{\footnotesize Reference category for the dependent variable is `McCain Vote'} \\
%\multicolumn{2}{l}{\footnotesize Sample limited to those that voted for either Obama or McCain} \\
%\multicolumn{2}{l}{\footnotesize \hspace{2pt} in the 2008 election} \\
%\multicolumn{2}{l}{\footnotesize All variables scaled to range between 0 and 1}
%\end{tabular} 
%\end{table}



\clearpage
\printbibliography
\end{refsection}
\end{appendices}










\end{document}∫